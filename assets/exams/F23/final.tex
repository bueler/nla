\documentclass[12pt]{amsart}
\addtolength{\topmargin}{-0.6in} % usually -0.25in
\addtolength{\textheight}{1.1in} % usually 1.25in
\addtolength{\oddsidemargin}{-0.7in}
\addtolength{\evensidemargin}{-0.7in}
\addtolength{\textwidth}{1.5in} %\setlength{\parindent}{0pt}

\newcommand{\normalspacing}{\renewcommand{\baselinestretch}{1.04}\tiny\normalsize}

\normalspacing

% macros
\usepackage{amssymb,xspace}
\usepackage{tikz}
\usepackage[pdftex,colorlinks=true]{hyperref}


\newtheorem*{thm}{Theorem}
\newtheorem*{lem}{Lemma}

\newcommand{\mtt}{\texttt}
\newcommand{\mtl}[1]{{\texttt{>>#1}}}
\usepackage{alltt}
\usepackage{fancyvrb}

\newcommand{\bu}{\mathbf{u}}
\newcommand{\bv}{\mathbf{v}}

\newcommand{\CC}{{\mathbb{C}}}
\newcommand{\RR}{{\mathbb{R}}}
\newcommand{\ZZ}{{\mathbb{Z}}}
\newcommand{\ZZn}{{\mathbb{Z}}_n}
\newcommand{\NN}{{\mathbb{N}}}

\newcommand{\eps}{\epsilon}
\newcommand{\grad}{\nabla}
\newcommand{\lam}{\lambda}
\newcommand{\ip}[2]{\mathrm{\left<#1,#2\right>}}
\newcommand{\erf}{\operatorname{erf}}

\renewcommand{\Re}{\operatorname{Re}}
\renewcommand{\Im}{\operatorname{Im}}
\newcommand{\Arg}{\operatorname{Arg}}

\newcommand{\Span}{\operatorname{span}}
\newcommand{\rank}{\operatorname{rank}}
\newcommand{\range}{\operatorname{range}}
\newcommand{\trace}{\operatorname{tr}}
\newcommand{\Null}{\operatorname{null}}

\newcommand{\Matlab}{\textsc{Matlab}\xspace}
\newcommand{\Octave}{\textsc{Octave}\xspace}
\newcommand{\pylab}{\textsc{pylab}\xspace}
\newcommand{\longMOP}{\textsc{Matlab}\big|\textsc{Octave}\big|\textsc{pylab}\xspace}
\newcommand{\MOP}{\textsc{M}\big|\textsc{O}\big|\textsc{p}\xspace}

\newcommand{\prob}[1]{\bigskip\noindent\large\textbf{#1.} \normalsize}
\newcommand{\bookprob}[1]{\bigskip\noindent\large\textbf{Exercise #1.} \normalsize}
\newcommand{\probpart}[1]{\smallskip\noindent\textbf{(#1)}\quad }
\newcommand{\aprobpart}[1]{\textbf{(#1)}\quad }


\newcommand*\circled[1]{\tikz[baseline=(char.base)]{
            \node[shape=circle,draw,inner sep=2pt] (char) {#1};}}


\begin{document}
\scriptsize \noindent Math 614 Numerical Linear Algebra (Bueler)
\thispagestyle{empty}

\bigskip
\Large\textbf{\centerline{Final Exam: analyze 3 algorithms}}

\medskip
\large\textbf{\centerline{Wednesday, 13 December, 1:00pm--3:00pm, Chapman 107}}

\normalsize
\bigskip
The in-class Final Exam will be a bit different, but fairly short and with a clear path for preparation.  I would like you to \textbf{summarize and analyze three major algorithms} which we have seen during the semester.  The allowed options are listed below.

\smallskip
You will have this document in your hand when you do the final.  Otherwise you must remember what you want to say.  \textbf{You may NOT bring your own notes, or the textbook,} to the Exam.  However, you are \textbf{strongly encouraged to draft and practice} your planned summaries.  Make sure to read relevant sections of the textbook.  Get feedback on your drafts from other students/faculty/friends/family/pets, or me.  Decide in advance how you will remember enough detail so as to recreate the summary during the Exam itself.

\smallskip
What should your summary and analysis include?  Clearly state the problem it solves.  (This includes what properties the input and outputs have.)  State the algorithm at a high level.  (A detailed pseudocode is not required, but either sketch the algorithm's steps in words or go ahead and give a pseudocode.)  What are the main geometric and/or algebraic ideas behind the algorithm?  What is the work estimate?  (Use big-O appropriately, and give leading-order constants if relevant.)  If there is a backward stability theorem, summarize what it says.  If it is important to compare to other well-known algorithms, please do so.

\smallskip
This is a writing assignment.  Please give an ``executive summary,'' making sure to point-out all the key ideas.  You may regard your target audience as a typical student in the first week of this course.  I will grade for correctness, clarity, and completeness.

\medskip

\newcommand{\alg}[1]{ \hfill \mbox{\emph{Alg.~#1}} }

\noindent\textbf{A: essential matrix factorizations}

\renewcommand{\labelenumi}{A\arabic{enumi}.}
\begin{enumerate}
\setlength{\itemsep}{4pt}
\item Householder QR factorization \alg{10.1}
\item Gauss elimination with partial pivoting \alg{21.1}
\end{enumerate}

\noindent\textbf{B: least squares solvers}

\renewcommand{\labelenumi}{B\arabic{enumi}.}
\begin{enumerate}
\setlength{\itemsep}{4pt}
\item least squares via normal equations \alg{11.1}
\item least squares via QR factorization \alg{11.2}
\item least squares via SVD \alg{11.3}
\end{enumerate}

\noindent\textbf{C: eigenvalue computations}

\renewcommand{\labelenumi}{C\arabic{enumi}.}
\begin{enumerate}
\setlength{\itemsep}{4pt}
\item reduction of Hermitian matrix to tridiagonal \alg{26.1}
\item Rayleigh quotient iteration \alg{27.1}
\end{enumerate}

\bigskip
\noindent \hrulefill

\noindent \textbf{\large Instructions:}

\begin{itemize}
\item \textbf{Please choose one algorithm from each category A,B,C above.}
\item \textbf{Write each summary on a separate sheet.  Identify each by ``A1'', etc.}
\item \textbf{Each summary should be more than half a page but at most one page; between 250 and 500 words.}
\end{itemize}
\end{document}


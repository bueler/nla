\documentclass[12pt]{amsart}
%prepared in AMSLaTeX, under LaTeX2e
\addtolength{\oddsidemargin}{-.8in}
\addtolength{\evensidemargin}{-.8in}
\addtolength{\topmargin}{-.8in}
\addtolength{\textwidth}{1.4in}
\addtolength{\textheight}{1.0in}
\newcommand{\normalspacing}{\renewcommand{\baselinestretch}{1.1}
        \tiny\normalsize}

\newtheorem*{thm}{Theorem}
\newtheorem*{defn}{Definition}
\newtheorem*{example}{Example}
\newtheorem*{problem}{Problem}
\newtheorem*{remark}{Remark}

\usepackage{amssymb,verbatim,alltt,xspace,hyperref}
\newcommand{\mtt}{\texttt}

\usepackage{fancyvrb}
\newcommand{\mfile}[1]{
\begin{quote}
\bigskip
\VerbatimInput[frame=single,label=\fbox{\normalsize \textsl{\,#1\,}},fontfamily=courier,fontsize=\small]{#1}
\end{quote}
}

\usepackage[final]{graphicx}
\newcommand{\mfigure}[1]{\includegraphics[height=3in,
keepaspectratio=true]{#1.eps}}

% macros
\newcommand{\CC}{\mathbb{C}}
\newcommand{\Div}{\nabla\cdot}
\newcommand{\eps}{\epsilon}
\newcommand{\grad}{\nabla}
\newcommand{\ZZ}{\mathbb{Z}}
\newcommand{\ip}[2]{\ensuremath{\left<#1,#2\right>}}
\newcommand{\lam}{\lambda}
\newcommand{\lap}{\triangle}
\newcommand{\tb}{\textsc{Morton \& Mayers}}
\newcommand{\RR}{\mathbb{R}}
\newcommand{\pexer}[2]{\bigskip\noindent\textbf{#1.} Exercise #2 in \tb.}
\newcommand{\prob}[1]{\medskip\noindent\textbf{#1.} }
\newcommand{\apart}[1]{\quad \textbf{(#1)} }
\newcommand{\epart}[1]{\medskip\noindent\textbf{(#1)} }
\newcommand{\note}[1]{[\scriptsize #1 \normalsize]}

\newcommand{\Matlab}{\textsc{Matlab}\xspace}
\newcommand{\Octave}{\textsc{Octave}\xspace}
\newcommand{\pylab}{\textsc{pylab}\xspace}
\newcommand{\MOP}{\textsc{MOP}\xspace}

\begin{document}
\scriptsize \noindent Math 615 Numerical Linear Algebra (Bueler) \hfill  Wednesday 13 December 2023
\normalsize\bigskip

\thispagestyle{empty}
\noindent\large\centerline{\textbf{Final Exam Algorithm Summaries: Evaluation}} \normalsize

\medskip
\noindent\centerline{\emph{Each Algorithm Summary is worth 25 points, for 50 total.}}

\bigskip
\noindent You will turn in two Case Summaries.  For each one I will indicate three numbers, as follows, which sum to a score out of 25.

\vspace{0.25in}

\noindent (\emph{10 pts}) \textbf{\underline{Completeness}, and staying on-topic}  
\vspace{0.3in}

\noindent (\emph{10 pts}) \textbf{\underline{Correctness{\footnotesize\strut}}, with reasonable notation}
\vspace{0.3in}

\noindent (\emph{5 pts}) \textbf{\underline{Readability} and clarity}
\vspace{0.35in}


\noindent \textbf{COMMENTS:}
\vfill

\hfill
Final Score: \underline{\phantom{dlkfj akfj}} + \underline{\phantom{dlkfj akfj}} $=$ \underline{\phantom{dlkfjadklja f sdakfj}} out of 50
\end{document}

\documentclass[12pt]{amsart}
\addtolength{\topmargin}{-0.6in} % usually -0.25in
\addtolength{\textheight}{1.1in} % usually 1.25in
\addtolength{\oddsidemargin}{-0.7in}
\addtolength{\evensidemargin}{-0.7in}
\addtolength{\textwidth}{1.5in} %\setlength{\parindent}{0pt}

\newcommand{\normalspacing}{\renewcommand{\baselinestretch}{1.04}\tiny\normalsize}

\normalspacing

% macros
\usepackage{amssymb,xspace}
\usepackage{tikz}
\usepackage[pdftex,colorlinks=true]{hyperref}


\newtheorem*{thm}{Theorem}
\newtheorem*{lem}{Lemma}

\newcommand{\mtt}{\texttt}
\newcommand{\mtl}[1]{{\texttt{>>#1}}}
\usepackage{alltt}
\usepackage{fancyvrb}

\newcommand{\bu}{\mathbf{u}}
\newcommand{\bv}{\mathbf{v}}

\newcommand{\CC}{{\mathbb{C}}}
\newcommand{\RR}{{\mathbb{R}}}
\newcommand{\ZZ}{{\mathbb{Z}}}
\newcommand{\ZZn}{{\mathbb{Z}}_n}
\newcommand{\NN}{{\mathbb{N}}}

\newcommand{\eps}{\epsilon}
\newcommand{\grad}{\nabla}
\newcommand{\lam}{\lambda}
\newcommand{\ip}[2]{\mathrm{\left<#1,#2\right>}}
\newcommand{\erf}{\operatorname{erf}}

\renewcommand{\Re}{\operatorname{Re}}
\renewcommand{\Im}{\operatorname{Im}}
\newcommand{\Arg}{\operatorname{Arg}}

\newcommand{\Span}{\operatorname{span}}
\newcommand{\rank}{\operatorname{rank}}
\newcommand{\range}{\operatorname{range}}
\newcommand{\trace}{\operatorname{tr}}
\newcommand{\Null}{\operatorname{null}}

\newcommand{\Matlab}{\textsc{Matlab}\xspace}
\newcommand{\Octave}{\textsc{Octave}\xspace}
\newcommand{\pylab}{\textsc{pylab}\xspace}
\newcommand{\longMOP}{\textsc{Matlab}\big|\textsc{Octave}\big|\textsc{pylab}\xspace}
\newcommand{\MOP}{\textsc{M}\big|\textsc{O}\big|\textsc{p}\xspace}

\newcommand{\prob}[1]{\bigskip\noindent\large\textbf{#1.} \normalsize}
\newcommand{\bookprob}[1]{\bigskip\noindent\large\textbf{Exercise #1.} \normalsize}
\newcommand{\probpart}[1]{\smallskip\noindent\textbf{(#1)}\quad }
\newcommand{\aprobpart}[1]{\textbf{(#1)}\quad }


\newcommand*\circled[1]{\tikz[baseline=(char.base)]{
            \node[shape=circle,draw,inner sep=2pt] (char) {#1};}}


\begin{document}
\scriptsize \noindent Math 614 Numerical Linear Algebra (Bueler) \hfill assigned: 1 December 2025
\thispagestyle{empty}

\bigskip
\Large\textbf{\centerline{Final Exam: Analyze Three Algorithms}}

\medskip
\large\textbf{\centerline{Thursday, 11 December, 1:00pm--3:00pm, Chapman 107}}

\normalsize
\bigskip
The in-class Final Exam will be a bit different, but fairly short and with a clear path for preparation.  I would like you to \textbf{summarize and analyze three major algorithms} which we have seen during the semester.

You will have this document in your hand during the Exam, but you must remember what you want to say!  \textbf{You may NOT bring notes, the textbook, or electronics of any kind} to the Exam.  Just bring a writing implement.

You are \textbf{strongly encouraged to draft and practice} your planned algorithm summaries.  Make sure to read relevant sections of the textbook.  Get feedback on your drafts from other students/friends/family/pets, or me.  Decide in advance how you will remember enough detail so as to recreate the summary during the Exam itself.

\textbf{Your summary and analysis should include:}  Clearly state the problem the algorithm solves, including what properties the input and outputs have.  State the algorithm at a high level.  What are the main geometric and/or algebraic ideas behind the algorithm?  A detailed pseudocode is not \emph{required}, but please either sketch the algorithm's steps in words or give a pseudocode.  What is the work estimate?  (Use big-O notation appropriately, and give leading-order constants if that is relevant.)  If there is a backward stability theorem, summarize what it says.  Compare to any other relevant/significant algorithms for the same or similar purposes.

\textbf{This is a writing assignment.}  Your goal is an ``executive summary,'' making sure to point-out all the key ideas.  Regard your target audience as a typical student in the first week of this course.  I will grade for correctness, completeness, and readability.

\bigskip
\noindent \hrulefill

\noindent \textbf{\large Instructions:}

\smallskip
Please choose one algorithm from each category A,B,C below, and summarize it.  Write each summary on a separate sheet.  Identify each by ``\textbf{A1}'', etc.~in the upper-left corner.  Put your name in the upper-right corner.  Each summary should be more than half a page but at most 1.5 pages.  It should be between 250 and 500 words; note that the text on this page above the horizontal line is 267 words.

\medskip

\newcommand{\alg}[1]{ \hfill \mbox{\emph{Alg.~#1}} }

\noindent A $=$ essential matrix factorizations

\renewcommand{\labelenumi}{\textbf{A\arabic{enumi}.}}
\begin{enumerate}
\setlength{\itemsep}{4pt}
\item Householder QR factorization \alg{10.1}
\item Gauss elimination with partial pivoting \alg{21.1}
\end{enumerate}

\noindent B $=$ least squares solvers

\renewcommand{\labelenumi}{\textbf{B\arabic{enumi}.}}
\begin{enumerate}
\setlength{\itemsep}{4pt}
\item solve least squares via normal equations \alg{11.1}
\item solve least squares via QR factorization \alg{11.2}
\item solve least squares via SVD \alg{11.3}
\end{enumerate}

\noindent C $=$ eigenvalue computations

\renewcommand{\labelenumi}{\textbf{C\arabic{enumi}.}}
\begin{enumerate}
\setlength{\itemsep}{4pt}
\item reduction of a Hermitian matrix to tridiagonal \alg{26.1}
\end{enumerate}

\end{document}


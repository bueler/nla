\documentclass[12pt]{article}

% Layout.
\usepackage[top=1.2in, bottom=0.9in, left=1in, right=1in, headheight=1in, headsep=6pt]{geometry}

% Fonts.
\usepackage{mathptmx}
\usepackage[scaled=1.0]{helvet}
\renewcommand{\emph}[1]{\textsf{\textbf{#1}}}

% Misc packages.
\usepackage{amsmath,amssymb,latexsym}
\usepackage{graphicx,hyperref}
\usepackage{array}
\usepackage{xcolor}
\usepackage{multicol}
\usepackage{tabularx,colortbl}
\usepackage{enumitem}

\hypersetup{
    colorlinks=true,
    linkcolor=blue,
    filecolor=magenta,      
    urlcolor=blue,
    pdfauthor={Ed Bueler}
    pdftitle={Syllabus for MATH F614 Fall 2023},
    }

% Paragraph spacing
\parindent 0pt
\parskip 6pt plus 1pt
\def\tableindent{\hskip 0.5 in}
\def\ts{\hskip 1.5 em}

\usepackage{fancyhdr}
\pagestyle{fancy} 
%\chead{\large\sf\textbf{}}
\lhead{\large\sf\textbf{Syllabus MATH F614}}
\rhead{\large\sf\textbf{Fall 2023}}
  
\newcommand{\localhead}[1]{\par\smallskip\textbf{#1} \smallskip\nobreak\\}%
\def\heading#1{\localhead{\large\emph{#1}}}
\def\subheading#1{\localhead{\emph{#1}}}

\newenvironment{clist}%
{\bgroup\parskip 0pt\begin{list}{$\bullet$}{\partopsep 4pt\topsep 0pt\itemsep -2pt}}%
{\end{list}\egroup}%


\begin{document}

\strut\par\vskip-12pt
\heading{Essential Information}

\vskip -12pt
\strut\hbox to \hsize{\tableindent\vtop{\halign{#\hfill\ts&#\hfil\cr
{\emph{Course Title}} & {\Large Numerical Linear Algebra} \cr
\strut & \cr
{\emph{Instructor}} & Ed Bueler \quad \href{mailto:elbueler@alaska.edu}{\texttt{elbueler\@@alaska.edu}} \cr
\strut & \cr
{\emph{Class meeting}} & MWF 2:15--3:15 pm, Chapman 107 \cr
\strut & \cr
{\emph{CRNs}} & in-person:\, 75513 \quad online: 75512 \href{https://canvas.alaska.edu/courses/15800}{(find zoom link on Canvas)}\cr
\strut & \cr
{\emph{Public website}} & \href{https://bueler.github.io/nla/}{\texttt{bueler.github.io/nla}}\cr
\strut & \cr
{\emph{Canvas website}} & \href{https://canvas.alaska.edu/courses/15800}{\texttt{canvas.alaska.edu/courses/15800}} \cr
\strut & \cr
\emph{Required text} & L.~N.~Trefethen and D.~Bau, \textsl{Numerical Linear Algebra}, \cr
 & SIAM Press 1997 \cr
}
\hfil}}


\heading{Description}
This course covers how matrices and vectors are actually handled in a fast and accurate manner on computers. This is essential technology for scientific and engineering computation.  Applications include solving large linear systems, least squares methods, systems of ordinary differential equations, inverse methods in geophysics, and Markov processes.  Numerical linear algebra is equally important for partial differential equations, network problems, and optimization, and it is an underlying theory in machine learning.

We will place linear algebra in a clear mathematical framework, emphasizing the geometry of the matrix action.  We will cover famous matrix decompositions and algorithms: singular value decomposition (SVD), Householder reflections and the QR decomposition, LU and Cholesky decompositions, spectral theorem, Schur decomposition, the QR method for eigenvalues, and Krylov methods.  Additionally, the conditioning of problems and the stability of floating-point algorithms are central themes.

Student competence with a scientific computing language is a course goal, so homework assignments will require actual implementations.  Examples/demonstrations in class, and also homework solutions, will routinely use Matlab/Octave, but student work can also be done in Python or Julia.  (All of these are well-suited to numerical linear algebra; see the separate document \textsl{Programming languages compared} (\href{https://bueler.github.io/compareMOP.pdf}{\texttt{bueler.github.io/compareMOP.pdf}}.)  The instructor and the textbook support Matlab/Octave, including with getting-started help.  If you have never used a programming language before then please show initiative in learning this aspect of the course.


\heading{Course Goals and Student Learning Outcomes}
At the end you will be able to understand and apply the ideas and algorithms of numerical linear algebra.  You will be very comfortable with a scientific computing language.


\clearpage \newpage
\phantom{foo}
\heading{Prerequisites}
In summary, the prerequisites are undergraduate linear algebra, exposure to or interest in scientific programming, and a certain amount of mathematical maturity.

Officially: \textsl{MATH 314 Linear Algebra or equivalent.  Recommended: MATH 421 Applied Analysis OR MATH 401 Introduction to Real Analysis OR equivalent post-calculus course in analysis.}


\heading{The Hybrid Classroom}
There are two sections of the class, in-person (901) and online (701).  They are treated as one course and occur simultaneously.  In this ``hybrid'' set-up, each lecture will be a recorded Zoom session generated from Chapman 107.  (The link for the Zoom session is \href{https://canvas.alaska.edu/courses/15800}{in Canvas}.  The recordings will be linked from inside Canvas only; they are not public.)  I will try to treat all students the same regarding proctored assessments---see below---and participation during class time.  Students have certain obligations to help make this work:
\begin{itemize}
\item \textbf{in-person students}: To allow in-class play with Matlab etc., and to help classroom communication for e.g.~group work, please bring a laptop if you can, and perhaps join the Zoom session so you can see the online students.  I prefer for in-person students to turn in their homework assignments on paper.
\item \textbf{online students}: Please sign into the Zoom session, from Canvas, just before class starts.  Please participate as energetically as you can, and, if possible, keep your camera on.  Regarding in-class group work, check for worksheet PDFs from the \href{https://bueler.github.io/nla/}{public site} before class starts.  When you turn in homework assignments electronically, please generate clear, well-ordered, and combined PDFs; this may require scanning documents.  You will need to schedule proctoring for the in-class assessments (see below), or attend in person on those days.
\end{itemize}


\heading{Schedule and Online Materials}
The \href{https://bueler.github.io/nla/}{public course website} includes a \href{https://bueler.github.io/nla/assets/general/F23/schedule.pdf}{day-by-day schedule} listing the textbook sections to be covered during each lecture, the due date of each homework Assignment, and the dates for the Midterm Quizzes and Final Exam.  Please consult this schedule frequently; it is subject to change and will be kept up to date.

Most course materials (syllabus, schedule, homework Assignments, code examples, etc.) will be posted on the \href{https://bueler.github.io/nla/}{public website}.  Some course materials (student grades, homework and exam solutions) will go on the \href{https://canvas.alaska.edu/courses/15800}{Canvas site}.


\heading{Office Hours and Communication}
My Office Hours are shown online at \href{http://bueler.github.io/OffHrs.htm}{\texttt{bueler.github.io/OffHrs.htm}}; I hold office hours in Chapman 306C.  Students can also schedule meetings with me outside of regular office hours.  I will use Canvas to send announcements.  If I need to contact you outside of class times, I'll try to email via Canvas.  (Please set your email address in Canvas to one that you check regularly!)


\clearpage\newpage
\phantom{foo}
\heading{Evaluation and Grades}
\vskip -10pt

\begin{tabular}{|c|c|c|}
\hline
Homework & nearly weekly & 50\% \\
\hline
Midterm Quiz 1 & in-class Wednesday 11 October  & 15\%  \\
\hline
Midterm Quiz 2 & in-class Wednesday 15 November & 15\%  \\
\hline
Final Exam     & in-class Wednesday 13 December, 1:00--3:00pm & 20\% \\
\hline
total & & 100\% \, \\
\hline
\end{tabular}

Scores for specific assessments may be adjusted based on the actual difficulty of the work and/or on average class performance, and adjustments will be applied to all students equally.  The scores of the various parts will be summed and the final course grade will be assigned as follows.

\begin{tabular}{llllll}
A  & 93--100\% & B- & 79--81\%  & D+ & 65--67\%  \\
A- & 90--92\%  & C+ & 76--78\%  & D  & 60--64\%  \\
B+ & 87--89\%  & C  & 68--75\%  & D- & 57--59\%  \\
B  & 82--86\%  & C- & not given & F  & $\le$ 56\%
\end{tabular}

These ranges are a guarantee and a lower bound.  I reserve the right to increase your grade above these ranges based on the actual difficulty of the work and/or on average class performance.  Any such increases will preserve grade ordering by weighted total score.


\heading{Homework}
Homework is due at the start of class.  \emph{Late homework is not accepted.}  If you have unavoidable circumstances which do not allow you to turn in an Assignment on time then please contact me (\href{mailto:elbueler@alaska.edu}{\texttt{elbueler\@@alaska.edu}}) in advance.

The homework consists of by-hand computations, design and analysis of numerical algorithms, computer implementation of those algorithms, by-hand and computer visualization, rigorously-justified examples and counter-examples, and proofs.  Problems very similar to, or shortened versions of, Homework problems will appear on the in-class Midterm Quizzes.

Exercises on the homework will require Matlab, or another suitable scientific computing language, both as a supercalculator and for writing programs.  Codes on homework solutions will only be in Matlab/Octave.  Homework assignments and their due dates will regularly be posted at the \href{https://bueler.github.io/nla/}{\texttt{public website}}.  The site also has a daily schedule of topics.  The schedule will be updated on an ongoing basis to reflect which topics were actually covered each day, so it is subject to change.  The public website will also link a growing list of short Matlab/Octave codes; this is a good resource for coding examples.


\heading{Exams}
There will be two in-class Midterm Quizzes covering mostly basic concepts and definitions.  Each of these will be 45 minutes, with a 15 minute debrief at the end after all exams are turned in.

The in-class Final Exam will require you to be familiar with \textbf{three} (updated 12/2/23) of the major methods we have studied.  I will describe the format of this Exam in more detail in due course.

Make-up Quizzes or Exam will be given only for documented extenuating circumstances, at my discretion.  Department policy (below) does not allow me to move the time of the Final Exam.


\clearpage\newpage
\phantom{foo}
\heading{Rules and Policies}
\vskip -20pt

\subheading{Incomplete Grade} 
Incomplete (I) will only be given in
  DMS courses in cases where
  the student has completed the majority (normally all but the last
  three weeks) of a course with a grade of C or better, but for
  personal reasons beyond his/her control has been unable to complete
  the course during the regular term. Negligence or indifference are
  not acceptable reasons for granting an incomplete grade.

\subheading{Late Withdrawals} 
A withdrawal after the deadline from a DMS course will
  normally be granted only in cases where the student is performing
  satisfactorily (i.e., C or better) in a course, but has exceptional
  reasons, beyond his/her control, for being unable to complete the
  course.  These exceptional reasons should be detailed in writing to
  the instructor, Department Chair and the Dean.

\subheading{No Early Final Examinations}
Final examinations for DMS courses shall not be held earlier than the date and time published in the official term schedule.  Normally, a student will not be allowed to take a final exam early.  Exceptions can be made by individual instructors, but should only be allowed in exceptional circumstances and in a manner which doesn't endanger the security of the exam.

\subheading{Academic Dishonesty}
Academic dishonesty, including cheating and plagiarism, will not be tolerated.  It is a violation of the Student Code of Conduct and will be punished according to UAF procedures.

\subheading{Student protections and service statement}
Every qualified student is welcome in my classroom.  As needed, I am happy to work with you, Disability Services, Veterans' Services, Rural Student Services, and so on, to find reasonable accommodations.  Students at this University are protected against sexual harassment and discrimination (Title IX), and minors have additional protections.  For more information on your rights as a student and the resources available to you to resolve problems, please go the following site: \href{https://www.uaf.edu/handbook/}{\texttt{www.uaf.edu/handbook}}.

\hfill  \scriptsize [syllabus version: \today] \normalsize

\end{document}

\documentclass[12pt]{article}

% Layout.
\usepackage[top=1.2in, bottom=0.9in, left=1in, right=1in, headheight=1in, headsep=6pt]{geometry}

% Fonts.
\usepackage{mathptmx}
\usepackage[scaled=1.0]{helvet}
\renewcommand{\emph}[1]{\textsf{\textbf{#1}}}

% Misc packages.
\usepackage{amsmath,amssymb,latexsym}
\usepackage{graphicx,hyperref}
\usepackage{array}
\usepackage{xcolor}
\usepackage{multicol}
\usepackage{tabularx,colortbl}
\usepackage{enumitem}

\hypersetup{
    colorlinks=true,
    linkcolor=blue,
    filecolor=magenta,      
    urlcolor=blue,
    pdftitle={Syllabus for MATH F614 Fall 2023 (Bueler)},
    pdfpagemode=FullScreen,
    }

% Paragraph spacing
\parindent 0pt
\parskip 6pt plus 1pt
\def\tableindent{\hskip 0.5 in}
\def\ts{\hskip 1.5 em}

\usepackage{fancyhdr}
\pagestyle{fancy} 
%\chead{\large\sf\textbf{}}
\lhead{\large\sf\textbf{Syllabus MATH F614}}
\rhead{\large\sf\textbf{Fall 2023}}
  
\newcommand{\localhead}[1]{\par\smallskip\textbf{#1} \smallskip\nobreak\\}%
\def\heading#1{\localhead{\large\emph{#1}}}
\def\subheading#1{\localhead{\emph{#1}}}

\newenvironment{clist}%
{\bgroup\parskip 0pt\begin{list}{$\bullet$}{\partopsep 4pt\topsep 0pt\itemsep -2pt}}%
{\end{list}\egroup}%


\begin{document}

\strut\par\vskip-12pt
\heading{Essential Information}

\vskip -12pt
\strut\hbox to \hsize{\tableindent\vtop{\halign{#\hfill\ts&#\hfil\cr
{\emph{Course Title}} & {\Large Numerical Linear Algebra} \cr
\strut & \cr
{\emph{Instructor}} & Ed Bueler \quad \href{mailto:elbueler@alaska.edu}{\texttt{elbueler\@@alaska.edu}} \cr
\strut & \cr
{\emph{Class meeting}} & MWF 2:15--3:15 pm, Chapman 107 \cr
\strut & \cr
{\emph{CRNs}} & in-person:\, 75513 \quad online: 75512 \href{https://canvas.alaska.edu/courses/15800}{(zoom link on Canvas site)}\cr
\strut & \cr
{\emph{Public website}} & \href{https://bueler.github.io/nla/}{\texttt{bueler.github.io/nla}}\cr
\strut & \cr
{\emph{Canvas website}} & \href{https://canvas.alaska.edu/courses/15800}{\texttt{canvas.alaska.edu/courses/15800}} \cr
\strut & \cr
\emph{Prerequisites} & Undergraduate linear algebra.  Exposure to computer programming.\cr
\strut & \cr
\emph{Required text} & L.~N.~Trefethen and D.~Bau, \textsl{Numerical Linear Algebra}, \cr
 & SIAM Press 1997 \cr
}
\hfil}}

\heading{Description}
This course covers numerical approximations of partial and ordinary differential equations on computers.  PDEs are the underlying structure for fluid flow, electric/magnetic fields, thermodynamics, elastic deformation, quantum mechanics and chemistry, and even financial mathematics.  ODEs are even more universal, sometimes as parts of PDE problems.  The course will include both linear and nonlinear examples.

The emphasis will be on finite difference methods for PDEs.  However, we will also understand ODE numerical schemes more deeply than from the brief introduction in an undergraduate differential equations course like MATH 302.

We will not be satisfied with seeing pretty pictures coming from the computer, so we will use mathematical analysis on numerical methods.  How well do our numbers correctly approximate the intended differential equation?  How are these methods implemented and  verified?  When are they stable?  Do we know in advance that they converge?

Homework assignments and a student-chosen project will require actual implementations in Matlab, Python, or another suitable language of your choice.  I will support Matlab but you may use other languages for assignments and the project.  Lectures will include Matlab demonstrations, plus getting-started help.  However, if you have not done it before then you will need to show initiative in learning numerical computation.


\heading{Course Goals and Student Learning Outcomes}
At the end of the course you will be able to evaluate and use numerical tools for solving scientific and engineering problems involving differential equations.  You will be able to code some of the basic methods and you will have the mathematical start needed for learning other approaches like the finite element method and spectral methods.  Student competence with actual scientific computing is a goal of the course; you will be comfortable using Matlab, Python, or a similar language for programming standard mathematial algorithms.


\heading{Schedule and Online Materials}
The \href{https://bueler.github.io/nade/}{public course website} includes a \href{https://bueler.github.io/nade/assets/general/S23/schedule.pdf}{schedule} listing the textbook sections to be covered during each lecture, the due date of each homework Assignment, and the dates for the Midterm and Final Exams.  Please consult this schedule frequently; it is subject to change and will be kept up to date.

Most course materials (syllabus, schedule, homework Assignments, code examples, project description, etc.) will be posted on the \href{https://bueler.github.io/nade/}{public website}.  Some course materials (student grades, homework and exam solutions, etc.) will go on the \href{https://canvas.alaska.edu/courses/13208}{Canvas site}.


\heading{Office Hours and Communication}
My Office Hours are shown online at \href{http://bueler.github.io/OffHrs.htm}{\texttt{bueler.github.io/OffHrs.htm}}; I hold office hours in Chapman 306C.  Students can also schedule meetings with me outside of regular office hours.  I will use Canvas to send announcements.  If I need to contact you outside of class times, I'll try to email via Canvas, so please set your email address in Canvas to one that you check regularly!

\medskip
\heading{Evaluation and Grades}
\vskip -10pt

\begin{tabular}{|c|c|c|}
\hline
Homework & nearly weekly & 50\% \\
\hline
Project: proposal & due Monday 3 April & 5\%  \\
\hline
Project: completed & due Monday 1 May & 15\%  \\
\hline
Midterm Exam & in-class Friday 24 March & 20\%  \\
\hline
Final Exam & \, in-class \emph{Thursday 4 May, 10:15--12:15} \, & 10\% \\
\hline
total & & 100\% \, \\
\hline
\end{tabular}

Scores for specific assessments may be adjusted based on the actual difficulty of the work and/or on average class performance, and adjustments will be applied to all students equally.  The scores of the various parts will be summed and the final course grade will be assigned as follows.

\begin{tabular}{llllll}
A  & 93--100\% & B- & 79--81\%  & D+ & 65--67\%  \\
A- & 90--92\%  & C+ & 76--78\%  & D  & 60--64\%  \\
B+ & 87--89\%  & C  & 68--75\%  & D- & 57--59\%  \\
B  & 82--86\%  & C- & not given & F  & $\le$ 56\%
\end{tabular}

These ranges are a guarantee and a lower bound.  I reserve the right to increase your grade above these ranges based on the actual difficulty of the work and/or on average class performance.  Any such increases will preserve grade ordering by weighted total score.


\heading{Homework}
Homework is due at the start of class.  \emph{Late homework is not accepted.}  If you have unavoidable circumstances which do not allow you to turn in an Assignment on time then please contact me (\href{mailto:elbueler@alaska.edu}{\texttt{elbueler\@@alaska.edu}}) in advance.

The homework consists of by-hand computations, design and analysis of numerical algorithms, computer implementation of those algorithms, by-hand and computer visualization, rigorously-justified examples and counter-examples, and some proofs.

Exercises on the homework will require Matlab, or another suitable language, both as a supercalculator and for writing programs.  Help will be given to learn Matlab, including examples in lecture.  Codes on homework solutions will only be in Matlab.  See the separate document \textsl{Programming languages compared} (\href{https://bueler.github.io/compareMOP.pdf}{\texttt{bueler.github.io/compareMOP.pdf}}) for other recommended scientific computing languages.

Homework assignments and their due dates will regularly be posted at the \href{https://bueler.github.io/nade/}{\texttt{public website}}.  The site also has a daily schedule of topics.  The schedule will be updated on an ongoing basis to reflect which topics were actually covered each day, so it is subject to change.  The public website will also link a growing list of short Matlab codes; this is a good resource for coding examples.

Problems very similar to, or shortened versions of, Homework problems will appear on the in-class Midterm Exam.


\heading{Project}
The project is in two parts, with the first part due midsemester and the second due just before final exams (dates above).  The topic will mostly be up to you, but I will make suggestions, and I reserve veto power on choice of topics.  The project must include both theory and practical computation.  A detailed handout will appear on Monday 27 March, outlining how you might choose a topic, and what are the expectations.


\heading{Exams}
There will be one in-class Midterm Exam covering mostly basic concepts and definitions. The in-class Final Exam will require you to be familiar with two of the methods we have studied.  A make-up Midterm will be given only for documented extenuating circumstances, at my discretion.  Department policy (below) does not allow me to move the time of the Final Exam.


\heading{Rules and Policies}
\vskip -20pt

\subheading{Incomplete Grade} 
Incomplete (I) will only be given in
  DMS courses in cases where
  the student has completed the majority (normally all but the last
  three weeks) of a course with a grade of C or better, but for
  personal reasons beyond his/her control has been unable to complete
  the course during the regular term. Negligence or indifference are
  not acceptable reasons for granting an incomplete grade.

\subheading{Late Withdrawals} 
A withdrawal after the deadline from a DMS course will
  normally be granted only in cases where the student is performing
  satisfactorily (i.e., C or better) in a course, but has exceptional
  reasons, beyond his/her control, for being unable to complete the
  course.  These exceptional reasons should be detailed in writing to
  the instructor, Department Chair and the Dean.

\subheading{No Early Final Examinations}
Final examinations for DMS courses shall not be held earlier than the date and time published in the official term schedule.  Normally, a student will not be allowed to take a final exam early.  Exceptions can be made by individual instructors, but should only be allowed in exceptional circumstances and in a manner which doesn't endanger the security of the exam.

\subheading{Academic Dishonesty}
Academic dishonesty, including cheating and plagiarism, will not be tolerated.  It is a violation of the Student Code of Conduct and will be punished according to UAF procedures.

\subheading{Student protections and service statement}
Every qualified student is welcome in my classroom.  As needed, I am happy to work with you, Disability Services, Veterans' Services, Rural Student Services, and so on, to find reasonable accommodations.  Students at this University are protected against sexual harassment and discrimination (Title IX), and minors have additional protections.  For more information on your rights as a student and the resources available to you to resolve problems, please go the following site: \href{https://www.uaf.edu/handbook/}{\texttt{www.uaf.edu/handbook}}.

\hfill  \scriptsize [syllabus version: \today] \normalsize

\end{document}

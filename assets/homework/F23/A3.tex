\documentclass[12pt]{amsart}
%prepared in AMSLaTeX, under LaTeX2e
\addtolength{\oddsidemargin}{-.65in} 
\addtolength{\evensidemargin}{-.65in}
\addtolength{\topmargin}{-.4in}
\addtolength{\textwidth}{1.3in}
\addtolength{\textheight}{.75in}

\renewcommand{\baselinestretch}{1.05}

\usepackage{verbatim} % for "comment" environment

\usepackage{palatino}

\newtheorem*{thm}{Theorem}
\newtheorem*{defn}{Definition}
\newtheorem*{example}{Example}
\newtheorem*{problem}{Problem}
\newtheorem*{remark}{Remark}

\newcommand{\mtt}{\texttt}
\usepackage{verbatim,xspace}

\usepackage[final]{graphicx}

% macros
\usepackage{amssymb}

\usepackage{hyperref}
\hypersetup{pdfauthor={Ed Bueler},
            pdfcreator={pdflatex},
            colorlinks=true,
            citecolor=blue,
            linkcolor=red,
            urlcolor=blue,
            }

\newcommand{\br}{\mathbf{r}}
\newcommand{\bv}{\mathbf{v}}
\newcommand{\bx}{\mathbf{x}}
\newcommand{\by}{\mathbf{y}}

\newcommand{\CC}{\mathbb{C}}
\newcommand{\RR}{\mathbb{R}}
\newcommand{\ZZ}{\mathbb{Z}}

\newcommand{\eps}{\epsilon}
\newcommand{\grad}{\nabla}
\newcommand{\lam}{\lambda}
\newcommand{\lap}{\triangle}

\newcommand{\ip}[2]{\ensuremath{\left<#1,#2\right>}}

\newcommand{\image}{\operatorname{im}}
\newcommand{\onull}{\operatorname{null}}
\newcommand{\rank}{\operatorname{rank}}
\newcommand{\range}{\operatorname{range}}
\newcommand{\trace}{\operatorname{tr}}

\newcommand{\prob}[1]{\bigskip\noindent\textbf{#1.}\quad }
\newcommand{\exer}[2]{\prob{Exercise #2 in Lecture #1}}

\newcommand{\pts}[1]{(\emph{#1 pts}) }
\newcommand{\epart}[1]{\medskip\noindent\textbf{(#1)}\quad }
\newcommand{\ppart}[1]{\,\textbf{(#1)}\quad }

\newcommand{\Matlab}{\textsc{Matlab}\xspace}
\newcommand{\Octave}{\textsc{Octave}\xspace}
\newcommand{\Python}{\textsc{Python}\xspace}
\newcommand{\Julia}{\textsc{Julia}\xspace}

\newcommand{\ds}{\displaystyle}


\begin{document}
\scriptsize \noindent Math 614 Numerical Linear Algebra (Bueler) \hfill \emph{assigned 12 September 2023; revised}
\normalsize\medskip

\Large\centerline{\textbf{Assignment 3}}
\large
\medskip

\centerline{\textbf{Due Friday 22 September 2023, at the start of class}}
\medskip
\normalsize

\thispagestyle{empty}

\bigskip
\noindent Please read Lectures 3,4,5,6 in the textbook \emph{Numerical Linear Algebra} by Trefethen and Bau.  This Assignment mostly covers norms (Lecture 3) and the SVD (Lectures 4 \& 5).

\bigskip
\noindent \textsc{Do the following exercises} from Lecture 2:

\begin{itemize}
\item \textbf{Exercise 2.3}
\end{itemize}

\bigskip
\noindent \textsc{Do the following exercises} from Lecture 3:

\begin{itemize}
\item \textbf{Exercise 3.2}
\item \textbf{Exercise 3.3}
\end{itemize}

\bigskip
\noindent \textsc{Do the following exercises} from Lecture 4:

\begin{itemize}
\item \textbf{Exercise 4.3}  \qquad \begin{minipage}[t]{0.68\textwidth}
Use the \texttt{svd} command on $A$.  Write a \Matlab (or other) \texttt{function} of the form \texttt{vismat(A)}.  Start by checking that the input matrix $A$ is in fact $2\times 2$, and that its entries are real.  Correctness of the program is more important than figure appearance.
\end{minipage}
\end{itemize}


\bigskip
\noindent \textsc{Do the following additional exercises.}

\prob{P7}  On page 21 of the textbook, equation (3.10) gives a formula for the $\infty$-norm of an $m\times n$ matrix.  Prove it:
    $$\|A\|_\infty = \max_{1\le i \le m} \|a_i^*\|_1.$$

\end{document}

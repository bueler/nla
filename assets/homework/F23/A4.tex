\documentclass[12pt]{amsart}
%prepared in AMSLaTeX, under LaTeX2e
\addtolength{\oddsidemargin}{-.65in} 
\addtolength{\evensidemargin}{-.65in}
\addtolength{\topmargin}{-.4in}
\addtolength{\textwidth}{1.3in}
\addtolength{\textheight}{.75in}

\renewcommand{\baselinestretch}{1.05}

\usepackage{verbatim} % for "comment" environment

\usepackage{palatino}

\newtheorem*{thm}{Theorem}
\newtheorem*{defn}{Definition}
\newtheorem*{example}{Example}
\newtheorem*{problem}{Problem}
\newtheorem*{remark}{Remark}

\newcommand{\mtt}{\texttt}
\usepackage{verbatim,xspace}

\usepackage[final]{graphicx}

% macros
\usepackage{amssymb}

\usepackage{hyperref}
\hypersetup{pdfauthor={Ed Bueler},
            pdfcreator={pdflatex},
            colorlinks=true,
            citecolor=blue,
            linkcolor=red,
            urlcolor=blue,
            }

\newcommand{\br}{\mathbf{r}}
\newcommand{\bv}{\mathbf{v}}
\newcommand{\bx}{\mathbf{x}}
\newcommand{\by}{\mathbf{y}}

\newcommand{\CC}{\mathbb{C}}
\newcommand{\RR}{\mathbb{R}}
\newcommand{\ZZ}{\mathbb{Z}}

\newcommand{\eps}{\epsilon}
\newcommand{\grad}{\nabla}
\newcommand{\lam}{\lambda}
\newcommand{\lap}{\triangle}

\newcommand{\ip}[2]{\ensuremath{\left<#1,#2\right>}}

\newcommand{\image}{\operatorname{im}}
\newcommand{\onull}{\operatorname{null}}
\newcommand{\rank}{\operatorname{rank}}
\newcommand{\range}{\operatorname{range}}
\newcommand{\trace}{\operatorname{tr}}

\newcommand{\prob}[1]{\bigskip\noindent\textbf{#1.}\quad }
\newcommand{\exer}[2]{\prob{Exercise #2 in Lecture #1}}

\newcommand{\pts}[1]{(\emph{#1 pts}) }
\newcommand{\epart}[1]{\medskip\noindent\textbf{(#1)}\quad }
\newcommand{\ppart}[1]{\,\textbf{(#1)}\quad }

\newcommand{\Matlab}{\textsc{Matlab}\xspace}
\newcommand{\Octave}{\textsc{Octave}\xspace}
\newcommand{\Python}{\textsc{Python}\xspace}
\newcommand{\Julia}{\textsc{Julia}\xspace}

\newcommand{\ds}{\displaystyle}


\begin{document}
\scriptsize \noindent Math 614 Numerical Linear Algebra (Bueler) \hfill \emph{assigned 20 September 2023; revised}
\normalsize\medskip

\Large\centerline{\textbf{Assignment 4}}
\large
\medskip

\centerline{\textbf{Due Monday 2 October 2023, at the start of class}}
\medskip
\normalsize

\thispagestyle{empty}

\bigskip
\noindent Please read Lectures 4,5,6,7 in the textbook \emph{Numerical Linear Algebra} by Trefethen and Bau.  This Assignment covers the SVD (Lectures 4 \& 5) and projectors (Lecture 6).

\bigskip
\noindent \textsc{Do the following exercise} from Lecture 4:

\begin{itemize}
\item \textbf{Exercise 4.4}
\end{itemize}

\bigskip
\noindent \textsc{Do the following exercises} from Lecture 5:

\begin{itemize}
\item \textbf{Exercise 5.1}
\item \textbf{Exercise 5.2}
\item \textbf{Exercise 5.3}
\end{itemize}

\bigskip
\noindent \textsc{Do the following exercises} from Lecture 6:

\begin{itemize}
\item \textbf{Exercise 6.1}
%\item \textbf{Exercise 6.3} \qquad \emph{You are free to use the SVD!}
\end{itemize}


\bigskip
\noindent \textsc{Do the following additional exercises.}

\prob{P8}  Use by-hand calculations to determine SVDs of the following matrices.  Note that in the decomposition $A = U \Sigma V^*$, the factor $\Sigma$ is unique but the factors $U$, $V$ are not.  Thus there will be more than one correct answer.  (\emph{Hints.  First, think.  Then use Theorem 5.4 or Theorem 5.5 if needed.  When in doubt, check (and show) that the unitary factors are indeed unitary.  Check that the singular values are in the proper order and that $A = U \Sigma V^*$ is actually true.})

\textbf{(a)} \quad $\ds \begin{bmatrix} -3 & 0 \\ 0 & 7 \end{bmatrix}$ \qquad
\textbf{(b)} \quad $\ds \begin{bmatrix} 0 & 0 \\ 0 & 3 \\ 0 & 0 \end{bmatrix}$ \qquad
\textbf{(c)} \quad $\ds \begin{bmatrix} 0 & 0 \\ 1 & 1 \end{bmatrix}$ \qquad
\textbf{(d)} \quad $\ds \begin{bmatrix} 2 & -1 \\ -1 & 2 \end{bmatrix}$

\prob{P9}  \ppart{a} Give an example of a projector which is not an orthogonal projector.

\epart{b} Show that if $P$ is a projector and $\lambda$ is an eigenvalue of $P$ then $\lambda = 0$ or $\lambda = 1$.

\epart{c} Show that if a projector is invertible then it is the identity.


\end{document}

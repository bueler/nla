\documentclass[12pt]{amsart}
%prepared in AMSLaTeX, under LaTeX2e
\addtolength{\oddsidemargin}{-.65in} 
\addtolength{\evensidemargin}{-.65in}
\addtolength{\topmargin}{-.4in}
\addtolength{\textwidth}{1.3in}
\addtolength{\textheight}{.75in}

\renewcommand{\baselinestretch}{1.05}

\usepackage{verbatim} % for "comment" environment

\usepackage{palatino}

\newtheorem*{thm}{Theorem}
\newtheorem*{defn}{Definition}
\newtheorem*{example}{Example}
\newtheorem*{problem}{Problem}
\newtheorem*{remark}{Remark}

\usepackage{fancyvrb,xspace}

\usepackage[final]{graphicx}

% macros
\usepackage{amssymb}

\usepackage{hyperref}
\hypersetup{pdfauthor={Ed Bueler},
            pdfcreator={pdflatex},
            colorlinks=true,
            citecolor=blue,
            linkcolor=red,
            urlcolor=blue,
            }

\newcommand{\br}{\mathbf{r}}
\newcommand{\bv}{\mathbf{v}}
\newcommand{\bx}{\mathbf{x}}
\newcommand{\by}{\mathbf{y}}

\newcommand{\CC}{\mathbb{C}}
\newcommand{\RR}{\mathbb{R}}
\newcommand{\ZZ}{\mathbb{Z}}

\newcommand{\eps}{\epsilon}
\newcommand{\grad}{\nabla}
\newcommand{\lam}{\lambda}
\newcommand{\lap}{\triangle}

\newcommand{\ip}[2]{\ensuremath{\left<#1,#2\right>}}

\newcommand{\image}{\operatorname{im}}
\newcommand{\onull}{\operatorname{null}}
\newcommand{\rank}{\operatorname{rank}}
\newcommand{\range}{\operatorname{range}}
\newcommand{\trace}{\operatorname{tr}}

\newcommand{\prob}[1]{\bigskip\noindent\textbf{#1.}\quad }
\newcommand{\exer}[2]{\prob{Exercise #2 in Lecture #1}}

\newcommand{\pts}[1]{(\emph{#1 pts}) }
\newcommand{\epart}[1]{\medskip\noindent\textbf{(#1)}\quad }
\newcommand{\ppart}[1]{\,\textbf{(#1)}\quad }

\newcommand{\Matlab}{\textsc{Matlab}\xspace}
\newcommand{\Octave}{\textsc{Octave}\xspace}
\newcommand{\Python}{\textsc{Python}\xspace}
\newcommand{\Julia}{\textsc{Julia}\xspace}

\newcommand{\ds}{\displaystyle}

\DefineVerbatimEnvironment{mVerb}{Verbatim}{numbersep=2mm,
frame=lines,framerule=0.1mm,framesep=2mm,xleftmargin=4mm,fontsize=\footnotesize}



\begin{document}
\scriptsize \noindent Math 614 Numerical Linear Algebra (Bueler) \hfill \emph{assigned 18 October 2023}
\normalsize\medskip

\Large\centerline{\textbf{Assignment 7}}
\large
\medskip

\centerline{\textbf{Due Monday 30 October 2023, at the start of class}}
\medskip
\normalsize

\thispagestyle{empty}

\bigskip
\noindent Please read Lectures 11,12,13,14,15 in the textbook \emph{Numerical Linear Algebra} by Trefethen and Bau.  This Assignment covers least squares, conditioning, and floating point.

\bigskip
\noindent \textsc{Do the following exercises} from Lecture 11:

\begin{itemize}
\item \textbf{Exercise 11.3}
\end{itemize}

\bigskip
\noindent \textsc{Do the following exercises} from Lecture 13:

\begin{itemize}
\item \textbf{Exercise 13.2} \quad \emph{Do parts \emph{(a)} and \emph{(b)} only.}
\end{itemize}


\bigskip
\noindent \textsc{Do the following additional exercises.}

\medskip

\prob{P13}  Suppose $A$ is a $100\times 100$ matrix with $\|A\|_2=20$ and $\|A\|_F=21$.  Give the sharpest possible lower bound on the 2-norm condition number of $A$.  (\emph{Hint.  Write everything in terms of singular values, and then think about best cases for $\kappa_2(A)$.})


\prob{P14}  For each problem, compute the absolute condition number $\hat\kappa$ and the relative condition number $\kappa$; generally both formulas will involve $x$.\footnote{You can use formulas (12.3) and (12.6) without justification.}  Choose the most convenient norm, but make your choice explicit.\footnote{For \textbf{a)} and \textbf{b)} just use absolute values for the norm.}  Comment on when the problem is well-conditioned or ill-conditioned; generally this answer also depends on $x$.
\renewcommand{\labelenumi}{\textbf{\alph{enumi})}}
\begin{enumerate}
\item Compute $x^3$ for $x>0$.
\item Compute $\cos x$ for real $x$.
\item For $x\in\CC^2$ compute $x_1x_2$, the product of the entries.
\item Fix $a\in\RR^m$, a column vector.  Compute the inner product $a^*x$ for $x\in \RR^m$.
\end{enumerate}


\prob{P15}  Consider the polynomial
\begin{align*}
p(x) &= (x-3)^{10} \\
     &= x^{10} - 30 x^9 + 405 x^8 - 3240 x^7 + 17010 x^6 - 61236 x^5 \\
     &\qquad   + 153090 x^4 - 262440 x^3 + 295245 x^2 - 196830 x + 59049
\end{align*}

\epart{a} Por $x=$ \small\verb|2.85:0.01:3.15|\normalsize, plot $p(x)$ by evaluating it via its coefficients $1,-30,405,\dots$

\epart{b} Plot $p(x)$ again on the same interval and same graph, using expression $(x-3)^{10}$.

\epart{c} In two or three sentences, compare and contrast the bad behavior here with the ill-conditioning phenomenon in Example 12.5 on page 92, i.e.~Wilkinson's example.


\prob{P16}  \emph{This is a reading assignment.  Actually read it!  It's good.}

\medskip
\noindent Please read the following 12 page encyclopedia entry:

\medskip
\begin{quote}
L.~N.~Trefethen, \emph{Numerical Analysis}, in W.~T.~Gowers, editor, Princeton Companion to Mathematics, Princeton U.~Press, 2008.

\href{https://people.maths.ox.ac.uk/trefethen/NAessay.pdf}{\texttt{people.maths.ox.ac.uk/trefethen/NAessay.pdf}}
\end{quote}

\medskip
\noindent Answer the following questions with a sentence or two at most:
\renewcommand{\labelenumi}{\emph{(\roman{enumi})}}
\begin{enumerate}
\item Give a one-sentence version of Trefethen's definition of ``numerical analysis.''
% from 2nd page: Numerical analysis is the study of algorithms for solving the problems of continuous mathematics, by which we mean problems involving real or complex variables.
\item Is analysis of rounding errors the main business of numerical analysis?  If not, what is?
\item Gaussian elimination with pivoting is a matrix factorization.  State it.
\item Trefethen refers to Householder triangularization, Algorithm 10.1 in our textbook, as ``QR factorization''.  But then what does the ``QR algorithm'' do?
%\item Which of the major ``algorithmic developments in the history of numerical analysis'' have we already covered in MATH 614?  Which do you think we will (or should) cover?
\item What is the ``central dogma'' of numerical linear algebra?
\item Fill the blank: ``The discovery of \underline{\phantom{LJpivotingAJ}} came quickly, but its theoretical analysis has proved astonishingly hard.'' %pivoting
\item What is the ``the biggest unsolved problem in numerical analysis''?
\end{enumerate}
\end{document}

\documentclass[12pt]{amsart}
%prepared in AMSLaTeX, under LaTeX2e
\addtolength{\oddsidemargin}{-.65in} 
\addtolength{\evensidemargin}{-.65in}
\addtolength{\topmargin}{-.4in}
\addtolength{\textwidth}{1.3in}
\addtolength{\textheight}{.85in}

\renewcommand{\baselinestretch}{1.05}

\usepackage{verbatim} % for "comment" environment

\usepackage{palatino}

\newtheorem*{thm}{Theorem}
\newtheorem*{defn}{Definition}
\newtheorem*{example}{Example}
\newtheorem*{problem}{Problem}
\newtheorem*{remark}{Remark}

\usepackage{fancyvrb,xspace,bm}

\usepackage[final]{graphicx}

% macros
\usepackage{amssymb}

\usepackage{hyperref}
\hypersetup{pdfauthor={Ed Bueler},
            pdfcreator={pdflatex},
            colorlinks=true,
            citecolor=blue,
            linkcolor=red,
            urlcolor=blue,
            }

\newcommand{\bbf}{\mathbf{f}}
\newcommand{\br}{\mathbf{r}}
\newcommand{\bs}{\mathbf{s}}
\newcommand{\bv}{\mathbf{v}}
\newcommand{\bx}{\mathbf{x}}
\newcommand{\by}{\mathbf{y}}

\newcommand{\bzero}{\bm{0}}

\newcommand{\CC}{\mathbb{C}}
\newcommand{\RR}{\mathbb{R}}
\newcommand{\ZZ}{\mathbb{Z}}

\newcommand{\eps}{\epsilon}
\newcommand{\grad}{\nabla}
\newcommand{\lam}{\lambda}
\newcommand{\lap}{\triangle}

\newcommand{\ip}[2]{\ensuremath{\left<#1,#2\right>}}

\newcommand{\image}{\operatorname{im}}
\newcommand{\onull}{\operatorname{null}}
\newcommand{\rank}{\operatorname{rank}}
\newcommand{\range}{\operatorname{range}}
\newcommand{\trace}{\operatorname{tr}}

\newcommand{\prob}[1]{\bigskip\noindent\textbf{#1.}\quad }
\newcommand{\exer}[2]{\prob{Exercise #2 in Lecture #1}}

\newcommand{\pts}[1]{(\emph{#1 pts}) }
\newcommand{\epart}[1]{\medskip\noindent\textbf{(#1)}\quad }
\newcommand{\ppart}[1]{\,\textbf{(#1)}\quad }

\newcommand{\Matlab}{\textsc{Matlab}\xspace}
\newcommand{\Octave}{\textsc{Octave}\xspace}
\newcommand{\Python}{\textsc{Python}\xspace}
\newcommand{\Julia}{\textsc{Julia}\xspace}

\newcommand{\fl}{\operatorname{fl}}

\newcommand{\ds}{\displaystyle}

\DefineVerbatimEnvironment{mVerb}{Verbatim}{numbersep=2mm,
frame=lines,framerule=0.1mm,framesep=2mm,xleftmargin=4mm,fontsize=\footnotesize}



\begin{document}
\scriptsize \noindent Math 614 Numerical Linear Algebra (Bueler) \hfill \emph{assigned 15 November 2023}
\normalsize\medskip

\Large\centerline{\textbf{Assignment 9}}
\large
\medskip

\centerline{\textbf{Due Wednesday 29 November 2023, at the start of class}}
\medskip
\normalsize

\thispagestyle{empty}

\bigskip
\noindent Please read Lectures 17,20,21,22,23 in the textbook \emph{Numerical Linear Algebra} by Trefethen and Bau.  (We are skipping Lectures 18 and 19.)  This Assignment primarily covers Gauss elimination and pivoting.

\bigskip
\noindent \textsc{Do the following exercises} from Lecture 17:

\begin{itemize}
\item \textbf{Exercise 17.2} \quad \emph{Hint.  Exactly what does Theorem 17.1 \underline{and Theorem  15.1} imply \dots}
\end{itemize}

\bigskip
\noindent \textsc{Do the following exercises} from Lecture 20:

\begin{itemize}
\item \textbf{Exercise 20.3} \quad \emph{Do part \emph{(a)} only.}
\item \textbf{Exercise 20.4}
\end{itemize}

\bigskip
\noindent \textsc{Do the following exercises} from Lecture 21:

\begin{itemize}
\item \textbf{Exercise 21.1}
\item \textbf{Exercise 21.5}  \qquad \begin{minipage}[t]{0.68\textwidth}  \emph{Feel free to only do the real case; you will get full credit.  In part \emph{(a)} I would advise going ahead and writing a pseudocode to ``describe'' it.  Then show a $3\times 3$ example for illustration?  In part \emph{(b)}, note that ``symmetric pivoting'' of $A$ is $PAP^\top$.}
\end{minipage}
\end{itemize}

\bigskip
\noindent \textsc{Do the following exercises} from Lecture 22:

\begin{itemize}
\item \textbf{Exercise 22.1}
\end{itemize}


\bigskip
\noindent \textsc{Do the following additional exercises.}

\medskip

\prob{P18}  This question requires nothing but calculus as a prerequisite.  It shows a major source of linear systems from applications.

\epart{a}  Consider these three equations, chosen for visualizability:
\begin{gather*}
x^2+y^2+z^2 = 4 \\
y = \cos(\pi z) \\
x = z^2
\end{gather*}
Sketch each equation individually as a surface in $\RR^3$.  (Do this by hand or in \Matlab.  Accuracy is not important.  The goal is to have a clear mental image of a nonlinear system as a set of intersecting surfaces.)  Considering where all three surfaces intersect, describe informally why there are two solutions, that is, two points $(x,y,z)\in\RR^3$ at which all three equations are satisfied.  Explain why both solutions are inside the closed box $0\le x \le 2, -1\le y \le 1, -2\le z \le 2$.

\epart{b}  Newton's method for a system of nonlinear equations is an iterative, approximate, and sometimes very fast, method for solving systems like the one above.

Let $\bx=(x_1,x_2,x_3)\in\RR^3$.  Suppose there are three scalar functions $f_i(\bx)$ forming a (column) vector function $\bbf(\bx)=(f_1,f_2,f_3)$, and consider the system
    $$\bbf(\bx)=\bzero.$$
(It is easy to put the part \textbf{(a)} system in this form.)  Let
	$$J_{ij} = \frac{\partial f_i}{\partial x_j}$$
be the Jacobian matrix: $J\in\RR^{3\times 3}$.  The Jacobian generally depends on location, i.e.~$J=J(\bx)$, and it generalizes the ordinary scalar derivative.

Newton's method itself is
\begin{gather}
J(\bx_n)\, \bs = - \bbf(\bx_n), \label{stepeqn} \\
\bx_{n+1} = \bx_n + \bs \label{updateeqn}
\end{gather}
where $\bs=(s_1,s_2,s_3)$ is the \emph{step} and $\bx_0$ is an initial iterate.  Equation \eqref{stepeqn} is a system of linear equations which determines $\bs$, and then equation \eqref{updateeqn} moves to the next iterate.

Using $\bx_0=(1,-1,1)$, write out equation \eqref{stepeqn} in the $n=0$ case, for the problem in part \textbf{(a)}, as a concrete linear system of three equations for the three unknown components of the step $\bs = (s_1,s_2,s_3)$.

\epart{c}  Implement Newton's method in \Matlab to solve the part \textbf{(a)} nonlinear system.  Show your script and generate at least five iterations.  Use $\bx_0=(1,-1,1)$ as an initial iterate to find one solution, and also find the other solution using a different initial iterate.  Note that \, \texttt{format long} \, is appropriate here for showing iterates.

\epart{d}  In calculus you likely learned Newton's method as a memorized formula, $x_{n+1} = x_n - f(x_n)/f'(x_n)$.  Rewrite equations \eqref{stepeqn}, \eqref{updateeqn} for $\RR^1$ to derive this formula.

\end{document}

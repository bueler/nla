\documentclass[12pt]{amsart}
%prepared in AMSLaTeX, under LaTeX2e
\addtolength{\oddsidemargin}{-.65in} 
\addtolength{\evensidemargin}{-.65in}
\addtolength{\topmargin}{-.4in}
\addtolength{\textwidth}{1.3in}
\addtolength{\textheight}{.75in}

\renewcommand{\baselinestretch}{1.05}

\usepackage{verbatim} % for "comment" environment

\usepackage{palatino}

\newtheorem*{thm}{Theorem}
\newtheorem*{defn}{Definition}
\newtheorem*{example}{Example}
\newtheorem*{problem}{Problem}
\newtheorem*{remark}{Remark}

\newcommand{\mtt}{\texttt}
\usepackage{verbatim,xspace}

\usepackage[final]{graphicx}

% macros
\usepackage{amssymb}

\usepackage{hyperref}
\hypersetup{pdfauthor={Ed Bueler},
            pdfcreator={pdflatex},
            colorlinks=true,
            citecolor=blue,
            linkcolor=red,
            urlcolor=blue,
            }

\newcommand{\br}{\mathbf{r}}
\newcommand{\bv}{\mathbf{v}}
\newcommand{\bx}{\mathbf{x}}
\newcommand{\by}{\mathbf{y}}

\newcommand{\CC}{\mathbb{C}}
\newcommand{\RR}{\mathbb{R}}
\newcommand{\ZZ}{\mathbb{Z}}

\newcommand{\eps}{\epsilon}
\newcommand{\grad}{\nabla}
\newcommand{\lam}{\lambda}
\newcommand{\lap}{\triangle}

\newcommand{\ip}[2]{\ensuremath{\left<#1,#2\right>}}

\newcommand{\image}{\operatorname{im}}
\newcommand{\onull}{\operatorname{null}}
\newcommand{\rank}{\operatorname{rank}}
\newcommand{\range}{\operatorname{range}}
\newcommand{\trace}{\operatorname{tr}}

\newcommand{\prob}[1]{\bigskip\noindent\textbf{#1.}\quad }
\newcommand{\exer}[2]{\prob{Exercise #2 in Lecture #1}}

\newcommand{\pts}[1]{(\emph{#1 pts}) }
\newcommand{\epart}[1]{\medskip\noindent\textbf{(#1)}\quad }
\newcommand{\ppart}[1]{\,\textbf{(#1)}\quad }

\newcommand{\Matlab}{\textsc{Matlab}\xspace}
\newcommand{\Octave}{\textsc{Octave}\xspace}
\newcommand{\Python}{\textsc{Python}\xspace}
\newcommand{\Julia}{\textsc{Julia}\xspace}

\begin{document}
\scriptsize \noindent Math 614 Numerical Linear Algebra (Bueler) \hfill \emph{Assigned FIXME}
\normalsize\medskip

\Large\centerline{\textbf{Assignment \#1}}
\large
\medskip

\centerline{\textbf{Due FIXME, at the start of class}}
\medskip
\normalsize

\thispagestyle{empty}

\bigskip

A major purpose of this assignment is to familiarize you with \Matlab (or \Octave, \Python, \Julia, etc.; see the \href{https://bueler.github.io/compareMOP.pdf}{\emph{Programming languages compared}} PDF).  You will need to find and download, or in any case set up, a copy of your preferred language.  Make sure you can use the command line, do basic operations at the command line (as a calculator), and plot simple things.  Make sure you can create a new program (i.e.~in a text file), save it, edit it, and run it at the command line.

\bigskip
\noindent Please read Lectures 1 and 2 in the textbook \emph{Numerical Linear Algebra} by Trefethen and Bau.

\bigskip
\noindent \textsc{Do the following exercises} from Lecture 1:

\begin{itemize}
\item \textbf{Exercise 1.3}
\end{itemize}


\bigskip
\noindent \textsc{Do the following additional exercises.}

\prob{P1}  (\emph{This exercise starts with a reminder of things you knew how to do during your undergraduate linear algebra course.  It is recommended that you find an undergrad linear algebra textbook, as it will be useful again this semester.})

Consider the $3\times 3$ real matrix
    $$A = \begin{bmatrix}   3     &     2    &     2 \\
                            -2    &     0    &     4 \\
                            1     &     1    &     2  \end{bmatrix}$$

\epart{a} \underline{By hand}, on paper:\begin{itemize}
\item compute the eigenvalues of $A$,
\item compute the determinant of $A$,
\item compute the rank of $A$,
\item compute the inverse of $A$ (if possible),
\item compute the inverse of $B=A(2:3,1:2)$ (if possible), and
\item solve the linear system $Ax=b$ where $b = \begin{bmatrix} -6 & 8 & -1 \end{bmatrix}^*$ (if possible).
\end{itemize}

\epart{b} Now check your work at the \Matlab command line.  (\emph{You'll use these \Matlab commands among others:} \texttt{eig,rank,det,inv,\,$\backslash$\;.})  As usual for any computational exercise, show the commands and their results in what you turn in.


% like Exercise 1.1
\prob{P2} Let $B$ be \emph{any} $4\times 4$ matrix to which we apply the following operations in turn:
\renewcommand{\labelenumi}{\arabic{enumi}.}
\begin{enumerate}
\item interchange columns 1 and 3
\item interchange rows 2 and 4
\item double column 2
\item add row 3 to row 1
\item subtract row 2 from each of the other rows
\item replace column 3 by column 4
\item delete row 2 (so the resulting matrix is $3\times 4$)
\end{enumerate}

\epart{a} Write the result as a product of eight matrices.

\epart{b} Write it again as a product $ABC$ of three matrices.  While $B$ remains arbitrary as before, please write specific matrices for $A$ and $C$.


\prob{P3}  (\emph{This problem exercises additional \Matlab commands such as} \texttt{rand,ones,norm, plot,loglog,semilogy}. \emph{It is also an exercise in communicating numerical results.  In fact, because you will generate 80 matrices, please do \emph{not} turn in a giant table of the matrices themselves!  Don't even turn in a table of their ranks, norms, or determinants.  Instead, use plots to communicate data and patterns.  It may make sense to compute averages over the 10 tries, but generally the data itself should appear in a plot.})

Write a \Matlab script which generates 10 random matrices of size $m\times m$ for each of these powers of two: $m=2,4,8,\dots,256$.  Every matrix will have entries which are independent random real numbers uniformly distributed on $[-5,5]$.  For each of these matrices compute the rank, the 2-norm, and the absolute value of the determinant.  Now communicate these data using plots in reasonable and appropriate ways.  Add the plot commands to your script.  A significant part of your script will be devoted to generating the plots.  In a few final sentences, describe the patterns you see.

\end{document}

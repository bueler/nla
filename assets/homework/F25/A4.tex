\documentclass[12pt,dvipsnames]{amsart}
%prepared in AMSLaTeX, under LaTeX2e
\addtolength{\oddsidemargin}{-.65in} 
\addtolength{\evensidemargin}{-.65in}
\addtolength{\topmargin}{-.4in}
\addtolength{\textwidth}{1.3in}
\addtolength{\textheight}{.8in}

\renewcommand{\baselinestretch}{1.05}

\usepackage{verbatim} % for "comment" environment
\usepackage{palatino}
\usepackage[final]{graphicx}
\usepackage{amssymb,enumitem,xspace,xcolor}
\usepackage{hyperref}
\hypersetup{pdfauthor={Ed Bueler},
            pdfcreator={pdflatex},
            colorlinks=true,
            citecolor=blue,
            linkcolor=red,
            urlcolor=blue,
            }

\newtheorem*{thm}{Theorem}
\newtheorem*{defn}{Definition}
\newtheorem*{example}{Example}
\newtheorem*{problem}{Problem}
\newtheorem*{remark}{Remark}

\newcommand{\br}{\mathbf{r}}
\newcommand{\bv}{\mathbf{v}}
\newcommand{\bx}{\mathbf{x}}
\newcommand{\by}{\mathbf{y}}

\newcommand{\CC}{\mathbb{C}}
\newcommand{\RR}{\mathbb{R}}
\newcommand{\ZZ}{\mathbb{Z}}

\newcommand{\eps}{\epsilon}
\newcommand{\grad}{\nabla}
\newcommand{\lam}{\lambda}
\newcommand{\lap}{\triangle}

\newcommand{\ip}[2]{\ensuremath{\left<#1,#2\right>}}

\newcommand{\image}{\operatorname{im}}
\newcommand{\onull}{\operatorname{null}}
\newcommand{\rank}{\operatorname{rank}}
\newcommand{\range}{\operatorname{range}}
\newcommand{\trace}{\operatorname{tr}}

\newcommand{\prob}[1]{\bigskip\noindent\textbf{#1.}\quad }
\newcommand{\exer}[2]{\prob{Exercise #2 in Lecture #1}}

\newcommand{\pts}[1]{(\emph{#1 pts}) }
\newcommand{\epart}[1]{\medskip\noindent\textbf{(#1)}\quad }
\newcommand{\ppart}[1]{\,\textbf{(#1)}\quad }

\newcommand{\Matlab}{\textsc{Matlab}\xspace}
\newcommand{\Octave}{\textsc{Octave}\xspace}
\newcommand{\Python}{\textsc{Python}\xspace}
\newcommand{\Julia}{\textsc{Julia}\xspace}

\newcommand{\ds}{\displaystyle}

\begin{document}
\scriptsize \noindent Math 614 Numerical Linear Algebra (Bueler) \hfill \emph{assigned 24 September 2025}
\normalsize\medskip

\Large\centerline{\textbf{Assignment \#4}}
\large
\medskip

\centerline{\textbf{Due Friday 3 October, at the start of class}}
\medskip
\normalsize

\thispagestyle{empty}

\bigskip

\noindent Please read Lectures 5, 6, 7, and 8 in the textbook \emph{Numerical Linear Algebra}, SIAM Press 1997, by Trefethen and Bau.

\bigskip
\noindent \textsc{Do the following exercises from the textbook}:

\begin{itemize}[itemsep=4pt]
\item \textbf{Exercise 5.2}
\item \textbf{Exercise 5.3}
\item \textbf{Exercise 6.1}
\item \textbf{Exercise 6.3} \quad You may use the SVD of $A$.
\item \textbf{Exercise 6.4}
\end{itemize}


\medskip
\noindent \textsc{Do the following additional problems.}

\prob{P10}  \ppart{a} Give an example of a projector which is not an orthogonal projector.

\epart{b} Show that if $P$ is a projector and $\lambda$ is an eigenvalue of $P$ then $\lambda = 0$ or $\lambda = 1$.

\epart{c} Show that if a projector is invertible then it is the identity.

\prob{P11}  Show that if $A\in \CC^{m\times m}$ with $\rank(A)=r$ then
  $$\frac{1}{\sqrt{r}} \|A\|_F \le \|A\|_2 \le \|A\|_F.$$
(\emph{Hint.}  You may use a theorem in Lecture 5.)  Note that computing $\|A\|_2$ requires $O(m^3)$ work (i.e.~number of arithmetic operations) while computing $\|A\|_F$ requires only $O(m^2)$ work.  Are there classes of matrices for which the inequalities above allow us to know the approximate size of $\|A\|_2$ with only the lesser amount of work?

\end{document}

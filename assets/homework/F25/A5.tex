\documentclass[12pt,dvipsnames]{amsart}
%prepared in AMSLaTeX, under LaTeX2e
\addtolength{\oddsidemargin}{-.65in} 
\addtolength{\evensidemargin}{-.65in}
\addtolength{\topmargin}{-.4in}
\addtolength{\textwidth}{1.3in}
\addtolength{\textheight}{.8in}

\renewcommand{\baselinestretch}{1.05}

\usepackage{fancyvrb}
\usepackage{palatino}
\usepackage[final]{graphicx}
\usepackage{amssymb,enumitem,xspace,xcolor}
\usepackage{hyperref}
\hypersetup{pdfauthor={Ed Bueler},
            pdfcreator={pdflatex},
            colorlinks=true,
            citecolor=blue,
            linkcolor=red,
            urlcolor=blue,
            }

\newtheorem*{thm}{Theorem}
\newtheorem*{defn}{Definition}
\newtheorem*{example}{Example}
\newtheorem*{problem}{Problem}
\newtheorem*{remark}{Remark}

\newcommand{\br}{\mathbf{r}}
\newcommand{\bv}{\mathbf{v}}
\newcommand{\bx}{\mathbf{x}}
\newcommand{\by}{\mathbf{y}}

\newcommand{\CC}{\mathbb{C}}
\newcommand{\RR}{\mathbb{R}}
\newcommand{\ZZ}{\mathbb{Z}}

\newcommand{\eps}{\epsilon}
\newcommand{\grad}{\nabla}
\newcommand{\lam}{\lambda}
\newcommand{\lap}{\triangle}

\newcommand{\ip}[2]{\ensuremath{\left<#1,#2\right>}}

\newcommand{\image}{\operatorname{im}}
\newcommand{\onull}{\operatorname{null}}
\newcommand{\rank}{\operatorname{rank}}
\newcommand{\range}{\operatorname{range}}
\newcommand{\trace}{\operatorname{tr}}

\newcommand{\prob}[1]{\bigskip\noindent\textbf{#1.}\quad }
\newcommand{\exer}[2]{\prob{Exercise #2 in Lecture #1}}

\newcommand{\pts}[1]{(\emph{#1 pts}) }
\newcommand{\epart}[1]{\medskip\noindent\textbf{(#1)}\quad }
\newcommand{\ppart}[1]{\,\textbf{(#1)}\quad }

\newcommand{\Matlab}{\textsc{Matlab}\xspace}
\newcommand{\Octave}{\textsc{Octave}\xspace}
\newcommand{\Python}{\textsc{Python}\xspace}
\newcommand{\Julia}{\textsc{Julia}\xspace}

\newcommand{\ds}{\displaystyle}

\DefineVerbatimEnvironment{mVerb}{Verbatim}{numbersep=2mm,
frame=lines,framerule=0.1mm,framesep=2mm,xleftmargin=4mm,fontsize=\footnotesize}


\begin{document}
\scriptsize \noindent Math 614 Numerical Linear Algebra (Bueler) \hfill \emph{assigned 3 October 2025}
\normalsize\medskip

\Large\centerline{\textbf{Assignment \#5}}
\large
\medskip

\centerline{\textbf{Due Monday 13 October, at the start of class}}
\medskip
\normalsize

\thispagestyle{empty}

\bigskip

\noindent Please read Lectures 7, 8, 10, and 11 in the textbook \emph{Numerical Linear Algebra}, SIAM Press 1997, by Trefethen and Bau.  The experiments in Lecture 9 are interesting, but not needed for any Homework or Exams.

\bigskip
\noindent \textsc{Do the following exercises from the textbook}:

\begin{itemize}[itemsep=4pt]
\item \textbf{Exercise 7.1}
\item \textbf{Exercise 7.3} \quad\, Start by explaining why $|\det(Q)|=1$ if $Q$ is unitary.
\item \textbf{Exercise 8.2} \quad\, \begin{minipage}[t]{0.68\textwidth}  Use your preferred language.  Implicit in this, and similar questions, is to show that your code works!  After the code, please show a brief command line session where you generate a generic/random matrix, run the code on it, and then verify that the outputs have the required properties.  Use \,\texttt{norm()} to avoid spewing numbers at me.  In other words, act like a professional.  See my previous Homework solutions for examples. \end{minipage}
\item \textbf{Exercise 10.2} \quad Same advice.
\item \textbf{Exercise 10.3}
\end{itemize}


\medskip
\noindent \textsc{Do the following additional problems.}

\prob{P12}  Suppose $A\in \CC^{m\times n}$, for $m\ge n$, is a matrix with ortho\emph{gonal}, but not necessarily ortho\emph{normal}, columns.  Describe its reduced QR decomposition.


\prob{P13}  \emph{While we have used QR to solve linear systems, here we see that QR factorization has a completely different application.  For more, see Lectures 24--29.}

\epart{a} By googling for ``unsolvable quintic polynomials'' or similar, confirm that there is a theorem which shows that fifth and higher-degree polynomials cannot be solved using finitely-many operations, including $n$th roots (``radicals'').  In other words, there is no finite formula for the solutions (``roots'') of such polynomial equations.  Who proved this theorem?  When?  Show a quintic polynomial for which it is known that there is no finite formula.  (\emph{You do \emph{not} need to prove that it is ``unsolvable''!})

%\clearpage\newpage
\epart{b} At the Matlab command line, do the following:
\begin{mVerb}
>> A = randn(5,5);  A = A' * A     % create a random 5x5 symmetric matrix
...
>> A0 = A;                         % save a copy of the original A
>> [Q, R] = qr(A);  A = R * Q
...                                % repeat about 10 times
>> [Q, R] = qr(A);  A = R * Q
\end{mVerb}
We start with a random, symmetric $5\times 5$ matrix $A_0$ and then generate a sequence of new matrices $A_i$.  Specifically, each matrix is factored and then the next matrix is generated by multiplying-back in reversed order:
    $$A_i = Q_i R_i \qquad \longrightarrow \qquad A_{i+1} = R_i Q_i.$$
What happens to the matrix entries when you iterate at least 10 times?    What do you observe about this sequence of $A_i$?  Now compare, both visually and using \texttt{norm()}, the vectors/lists \verb|sort(diag(A))| to \verb|sort(eig(A0))|.

\medskip
\noindent \emph{Without turning in anything for this stuff, also do:}
\begin{itemize}
\item \emph{Use a} \texttt{for} \emph{loop to see results from 100 iterations.}
\item \emph{Repeat the experiment for a $13\times 13$ matrix.  That is, there is nothing special about $5\times 5$.}
\end{itemize} 

\epart{c} To see a bit more of what is going on in part \textbf{(b)}, show that
    $$A_{i+1} = Q_i^* A_i Q_i.$$
So, at least this shows $A_{i+1}$ has exactly the same eigenvalues as $A_i$\,; \,explain why.

\epart{d} Apparently we have discovered an eigenvalue solver for symmetric matrices.  Write a few sentences which relate the context from part \textbf{(a)} to the results in \textbf{(b)} and \textbf{(c)}.  (\emph{Hint.  Think and speculate.  Then read ``A Fundamental Difficulty'' in Lecture 25 to confirm your understanding.})

\end{document}

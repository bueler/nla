\documentclass[11pt]{amsart}
%prepared in AMSLaTeX, under LaTeX2e
\addtolength{\oddsidemargin}{-.9in} 
\addtolength{\evensidemargin}{-.9in}
\addtolength{\topmargin}{-.9in}
\addtolength{\textwidth}{1.5in}
\addtolength{\textheight}{1.5in}

\renewcommand{\baselinestretch}{1.05}

\usepackage{verbatim} % for "comment" environment

\usepackage{palatino}

\usepackage[final]{graphicx}

\usepackage{tikz}
\usetikzlibrary{positioning}

\usepackage{amssymb,enumitem,xspace,fancyvrb,mathabx}

\newtheorem*{thm}{Theorem}
\newtheorem*{defn}{Definition}
\newtheorem*{example}{Example}
\newtheorem*{problem}{Problem}
\newtheorem*{remark}{Remark}

\DefineVerbatimEnvironment{mVerb}{Verbatim}{numbersep=2mm,frame=lines,framerule=0.1mm,framesep=2mm,xleftmargin=4mm,fontsize=\footnotesize}

% macros
\usepackage{amssymb}
\newcommand{\bA}{\mathbf{A}}
\newcommand{\bB}{\mathbf{B}}
\newcommand{\bE}{\mathbf{E}}
\newcommand{\bF}{\mathbf{F}}
\newcommand{\bJ}{\mathbf{J}}

\newcommand{\bb}{\mathbf{b}}
\newcommand{\br}{\mathbf{r}}
\newcommand{\bv}{\mathbf{v}}
\newcommand{\bw}{\mathbf{w}}
\newcommand{\bx}{\mathbf{x}}

\newcommand{\Div}{\ensuremath{\nabla\cdot}}
\newcommand{\Curl}{\ensuremath{\nabla\times}}
\newcommand{\eps}{\epsilon}
\newcommand{\grad}{\nabla}
\newcommand{\ip}[2]{\ensuremath{\left<#1,#2\right>}}
\newcommand{\lam}{\lambda}
\newcommand{\lap}{\triangle}

\newcommand{\Null}{\operatorname{null}}
\newcommand{\rank}{\operatorname{rank}}
\newcommand{\range}{\operatorname{range}}
\newcommand{\trace}{\operatorname{tr}}

\newcommand{\RR}{\mathbb{R}}
\newcommand{\ZZ}{\mathbb{Z}}

\newcommand{\ds}{\displaystyle}

\newcommand{\prob}[1]{\bigskip\noindent\textbf{#1.}\quad }
\newcommand{\epart}[1]{\bigskip\noindent\textbf{(#1)}\quad }
\newcommand{\ppart}[1]{\,\textbf{(#1)}\quad }

\newcommand{\fl}{\operatorname{fl}}
\newcommand{\emach}{\eps_{\text{machine}}}
\newcommand{\Matlab}{\textsc{Matlab}\xspace}


\begin{document}
\scriptsize \noindent Math 614 Numerical Linear Algebra (Bueler) \hfill \fbox{\emph{Not to be turned in!}}
\normalsize\medskip

\Large\centerline{\textbf{Worksheet: Proving backward stability.}}
\medskip
\normalsize

\thispagestyle{empty}

\bigskip
\noindent Recall axiom (13.5): for each $x\in\RR$ there is a real value $\eps$ so that $\fl(x)=x(1+\eps)$ and $|\eps|\le \emach$.  Recall axiom (13.7): for all $x,y\in\RR$, and for operations $\ast=+,-,\times,\div$, there is a real value $\eps$ so that $x \circledast y=(x\ast y)(1+\eps)$ and $|\eps|\le \emach$.

\medskip
\noindent A \emph{problem} is a function $f:X \to Y$, where $X$ and $Y$ are normed vector spaces.  An \emph{algorithm} is another function $\tilde f:X \to Y$.  On computers for which axioms (13.5) and (13.7) are true, we say the algorithm is \emph{backward stable} for some input $x$ if
    $$\boxed{\tilde f(x) = f(\tilde x)} \quad \text{for some } \tilde x \text{ satisfying } \quad\boxed{\ds \frac{\|\tilde x - x\|}{\|x\|} = O(\emach)}.$$

\medskip
\noindent In each of the following exercises, if I have not done it already, start by precisely identifying the problem and the algorithm.  Then prove backward stability.

\bigskip
\begin{enumerate}[leftmargin=15pt]
\renewcommand{\labelenumi}{\Large \textbf{\arabic{enumi}.}}
\item Prove that the obvious algorithm $\tilde f(x)=\fl(x_1) \odiv \fl(x_2)$ for dividing real numbers ($f(x)=x_1 \div x_2$) is backward stable.  Assume $x_2\ne 0$.

\vfill
\item Prove that the obvious algorithm for raising a real number to the third power is backward stable.

\vfill
\newpage
\item Let $x\in\RR^3$ be fixed.  Prove that the obvious computer algorithm for computing the inner product $a^* x$, for $a\in\RR^3$, is backward stable.

\vfill
\end{enumerate}

\end{document}

\documentclass[11pt]{amsart}
\addtolength{\topmargin}{-0.6in} % usually -0.25in
\addtolength{\textheight}{1.2in} % usually 1.25in
\addtolength{\oddsidemargin}{-0.7in}
\addtolength{\evensidemargin}{-0.7in}
\addtolength{\textwidth}{1.5in} %\setlength{\parindent}{0pt}

\newcommand{\normalspacing}{\renewcommand{\baselinestretch}{1.1}\tiny\normalsize}
\newcommand{\bigspacing}{\renewcommand{\baselinestretch}{1.21}\tiny\normalsize}
\newcommand{\tablespacing}{\renewcommand{\baselinestretch}{1.0}\tiny\normalsize}
\normalspacing

% macros
\usepackage{amssymb,xspace}
\usepackage[pdftex,colorlinks=true]{hyperref}

\usepackage[final]{graphicx}
\newcommand{\regfigure}[3]{\includegraphics[height=#2in,width=#3in]{#1.eps}}

\newtheorem*{thm}{Theorem}
\newtheorem*{lem}{Lemma}

\newcommand{\mtt}{\texttt}
\newcommand{\mtl}[1]{{\texttt{>>#1}}}
\usepackage{alltt}
\usepackage{fancyvrb}

\newcommand{\CC}{{\mathbb{C}}}
\newcommand{\RR}{{\mathbb{R}}}
\newcommand{\ZZ}{{\mathbb{Z}}}
\newcommand{\ZZn}{{\mathbb{Z}}_n}
\newcommand{\NN}{{\mathbb{N}}}
\newcommand{\eps}{\epsilon}
\newcommand{\lam}{\lambda}
\newcommand{\bu}{\mathbf{u}}
\newcommand{\bv}{\mathbf{v}}
\newcommand{\ip}[2]{\mathrm{\left<#1,#2\right>}}
\newcommand{\erf}{\operatorname{erf}}

\renewcommand{\Re}{\operatorname{Re}}

\newcommand{\Span}{\operatorname{span}}
\newcommand{\rank}{\operatorname{rank}}
\newcommand{\range}{\operatorname{range}}
\newcommand{\trace}{\operatorname{tr}}
\newcommand{\Null}{\operatorname{null}}

\newcommand{\Matlab}{\textsc{Matlab}\xspace}
\newcommand{\Octave}{\textsc{Octave}\xspace}
\newcommand{\pylab}{\textsc{pylab}\xspace}
\newcommand{\longMOP}{\textsc{Matlab}\big|\textsc{Octave}\big|\textsc{pylab}\xspace}
\newcommand{\MOP}{\textsc{M}\big|\textsc{O}\big|\textsc{p}\xspace}

\newcommand{\prob}[1]{\bigskip\noindent\large\textbf{#1.} \normalsize}
\newcommand{\bookprob}[1]{\bigskip\noindent\large\textbf{Exercise #1.} \normalsize}
\newcommand{\probpart}[1]{\smallskip\noindent\textbf{(#1)}\quad }
\newcommand{\aprobpart}[1]{\textbf{(#1)}\quad }

\newcommand{\textbook}{\textsc{Trefethen \& Bau}}

\begin{document}
\scriptsize \noindent Math 614 Numerical Linear Algebra (Bueler) \hfill \today
\thispagestyle{empty}

\bigskip
\Large\centerline{\textbf{Review Topics}}

\Large\centerline{for in-class \textbf{Midterm Quiz 1} on}

\Large\centerline{\textbf{Wednesday 11 October 2023}}

\normalsize
\bigskip
\begin{quote}
The Midterm Quiz will cover Lectures 1, 2, 3, 4, 5, 6, 7, 8 in \textbook.  The problems will be of these types: state definitions, state theorems, state algorithms (as pseudocode or \Matlab), describe or illustrate geometrical ideas, basic applications of theorems, quick calculations, prove simple theorems/corollaries.
\end{quote}
\bigskip

\bigspacing
\noindent \textbf{Definitions}. Know how to define:
\begin{itemize}
\item matrix-vector product; matrix-matrix product
\item adjoint; hermitian; transpose
\item inner product; outer product
\item unitary
\item $\|\cdot\|_p$ of a vector in $\CC^m$ for any $1\le p \le \infty$
\item induced matrix norm $\|\cdot\|$
\item Frobenius matrix norm $\|\cdot\|_F$
\item projector; orthogonal projector
\item eigenvalue; eigenvector
\item singular value
%\item residual (for overdetermined ``$Ax=b$'')
%\item absolute condition number $\hat\kappa(x)$ of a problem instance
%\item relative condition number $\kappa(x)$ of a problem instance
%\item condition number of a square matrix: $\kappa(A)=\|A\|\,\|A^{-1}\|$
\end{itemize}

\normalspacing

\medskip\noindent \textbf{Matrix Factorizations and Constructions}.  Know the properties of the factors in each factorization below.  (\emph{For example, $\hat U$ has orthonormal columns and is of same size as $A$ in the $m\ge n$ case of the reduced SVD factorization $A=\hat U \hat \Sigma V^*$.})  Assume $A\in \CC^{m\times n}$ unless otherwise stated.  Be able to use the factorization in simple calculations.  Be able to compute the factorization by hand in sufficiently small and simple cases.

\bigspacing

\begin{itemize}
\item full SVD: $A=U \Sigma V^*$
\item reduced SVD, $m\ge n$: $A=\hat U \hat \Sigma V^*$
\item full QR, $m\ge n$: $A=QR$
\item reduced QR, $m\ge n$: $A=\hat Q \hat R$
\item eigenvalue, $m=n$: $A = X\Lambda X^{-1}$ \hfill {\footnotesize\emph{$\leftarrow$ not always possible!}}
\item orthogonal projector onto $\range(A)$: $P=\hat Q\hat Q^* = A\left(A^* A\right)^{-1} A^*$
%\item Householder reflector: $F = I - 2 v v^* / (v^* v)$
\end{itemize}

\medskip\noindent \textbf{Algorithms}. Be able to state these algorithms, including the amount of work to leading order.
\begin{itemize}
\item matrix-vector and matrix-matrix products
\item Alg.~7.1: classical Gram-Schmidt for $QR$
\item Alg.~8.1: modified Gram-Schmidt for $QR$
\item high-level algorithm to solve $Ax=b$ when invertible $A=U\Sigma V^*$ has known SVD
\item high-level algorithm to solve $Ax=b$ when invertible $A=QR$ has known QR decomposition
%\item Alg.~10.1: Householder triangularization for $QR$
%\item Alg.~10.2: compute $Q^*b$ after Householder triangularization
%\item Alg.~11.2: $QR$ for least squares on overdetermined ``$Ax=b$''
%\item Alg.~11.1: normal equations for least squares on overdetermined ``$Ax=b$''
\end{itemize}

\newpage\noindent \textbf{Facts and Formulas}.  Know as facts.  Be able to prove unless otherwise stated.
\begin{itemize}
\item Cauchy-Schwarz: $|x^* y| \le \|x\|\,\|y\|$ \qquad [\emph{proof not required}]
\item invariance of $\|\cdot\|_2$ and $\|\cdot\|_F$ matrix norms under unitaries
\item $\|A\|_F = \sqrt{\sigma_1^2 + \dots + \sigma_r^2}$
\item $\|A\|_2 = \sigma_1$
\item for $A\in \CC^{m\times n}$ with $m\ge n$, $A$ has full rank if and only if $A^* A$ is nonsingular
\item $\rank(A)$ is number of nonzero singular values (in exact arithmetic)
\item if $m=n$ then $|\det(A)|=\prod_{i=1}^m \sigma_i$
\item the singular values of $A$ are the square roots of the eigenvalues of $A^*A$
\item if $P$ is an (orthogonal) projector then $I-P$ is an (orthogonal) projector
\item if $P$ is an orthogonal projector then $I-2P$ is unitary
%\item if $A$ is square and $\|\cdot\| = \|\cdot\|_2$:\quad $\kappa(A)=\sigma_1/\sigma_m$
\end{itemize}

\normalspacing

\bigskip\noindent \textbf{Ideas}.  Please be comfortable with the following ideas!  Some ideas correspond to theorems, but otherwise it is just a perspective or paradigm.

\bigspacing
\begin{itemize}
\item L1 and L2: how to think about $Ax$, $A^{-1}b$, $Qx$, $Q^* b$
\item L4: the image of the unit sphere under any $m\times n$ matrix is a hyperellipsoid
\item L5: sums like this are optimal (in what sense?) approximations of $A$:
	$$A_\nu = \sum_{j=1}^\nu \sigma_j u_j v_j^*$$
\item L6: given $A$, the orthogonal projector onto $\range(A)$ is constructable \dots know the formulas \dots full rank versus not full rank?
\item L7: construction of orthogonal functions (e.g.~orthogonal polynomials) is an application of the Gram-Schmidt process, and/or of $A=QR$, when the columns of $A$ are infinitely long
\item L8: the leading-order number of operations in some algorithm can be counted by geometric arguments
%\item L10: Gram-Schmidt is ``triangular orthogonalization'' while Householder is ``orthogonal triangularization''\footnote{Gauss elimination is ``triangular triangularization'' \dots but it will not be on the Midterm Exam!}
\end{itemize}


\end{document}


\documentclass[12pt]{amsart}
%prepared in AMSLaTeX, under LaTeX2e
\addtolength{\oddsidemargin}{-.55in} 
\addtolength{\evensidemargin}{-.55in}
\addtolength{\topmargin}{-.2in}
\addtolength{\textwidth}{1.0in}
\addtolength{\textheight}{.4in}

\renewcommand{\baselinestretch}{1.05}

\usepackage{verbatim,fancyvrb}

\usepackage{tikz,palatino,amssymb}

\newtheorem*{thm}{Theorem 16.0}
\newtheorem*{defn}{Definition}
\newtheorem*{example}{Example}
\newtheorem*{problem}{Problem}
\newtheorem*{remark}{Remark}

\newcommand{\mtt}{\texttt}
\usepackage{alltt,xspace}
\newcommand{\mfile}[1]
{\medskip\begin{quote}\scriptsize \begin{alltt}\input{#1.m}\end{alltt} \normalsize\end{quote}\medskip}

%\usepackage[final]{graphicx}

\usepackage[pdftex, colorlinks=true, plainpages=false, linkcolor=blue, citecolor=red, urlcolor=blue]{hyperref}

% macros
\newcommand{\br}{\mathbf{r}}
\newcommand{\bv}{\mathbf{v}}
\newcommand{\bx}{\mathbf{x}}
\newcommand{\by}{\mathbf{y}}

\newcommand{\CC}{\mathbb{C}}
\newcommand{\RR}{\mathbb{R}}
\newcommand{\ZZ}{\mathbb{Z}}

\newcommand{\eps}{\epsilon}
\newcommand{\grad}{\nabla}
\newcommand{\lam}{\lambda}
\newcommand{\lap}{\triangle}

\newcommand{\ip}[2]{\ensuremath{\left<#1,#2\right>}}

%\renewcommand{\det}{\operatorname{det}}
\newcommand{\onull}{\operatorname{null}}
\newcommand{\rank}{\operatorname{rank}}
\newcommand{\range}{\operatorname{range}}

\newcommand{\prob}[1]{\bigskip\noindent\textbf{#1.}\quad }
\newcommand{\exer}[2]{\prob{Exercise #2 in Lecture #1}}

\newcommand{\pts}[1]{(\emph{#1 pts}) }
\newcommand{\epart}[1]{\medskip\noindent\textbf{(#1)}\quad }
\newcommand{\ppart}[1]{\,\textbf{(#1)}\quad }

\newcommand{\Matlab}{\textsc{Matlab}\xspace}

\DefineVerbatimEnvironment{mVerb}{Verbatim}{numbersep=2mm,
frame=lines,framerule=0.1mm,framesep=2mm,xleftmargin=4mm,fontsize=\footnotesize}

\newcommand{\ema}{\emach}
\newcommand{\emach}{\eps_{\!_{\text{m}}}}


\begin{document}
\scriptsize \noindent Math 614 Numerical Linear Algebra (Bueler)
\normalsize

\medskip\bigskip
\Large
\centerline{Multiplication by a unitary matrix is backward-stable}

\medskip
\normalsize

\thispagestyle{empty}

This is an idea which I think should have been in the text\footnote{Trefethen \& Bau, \emph{Numerical Linear Algebra}, SIAM Press, 1997.} itself, and not just in Exercise 16.1(a).  Its proof uses an idea not seen in other ``show the algorithm is backward-stable'' arguments.  We start in an unexpected way, by bounding the forward error $\|\tilde f(A)-f(A)\|$.  Then the combination of unitarity and linearity allows us to transfer the forward error to a backward error $\|\tilde A - A\|$ using an input $\tilde A$ for which $\tilde f(A) = f(\tilde A)$.

\medskip
\begin{thm}
Fix $Q\in\CC^{m\times m}$ unitary.  On a computer satisfying (13.5) and (13.7), the obvious matrix-matrix multiplication algorithm is backward-stable for the problem
    $$f(A) = QA, \qquad A\in\CC^{m\times n}.$$
\end{thm}

\begin{proof}
Each entry of the product $QA$ is an inner product $g(y)=x^*y$.  The obvious algorithm for inner products is backward stable, so that $\tilde g(y) = g(\tilde y)$ where $\tilde y = y + \delta y$ with $\|\delta y\|_2 \le c(m) \ema \|y\|_2$ with some constant $c(m)$ independent of $y$ and $\ema$.

Consider the $i,j$ entry of the product $QA$.  To apply the above idea, let $x = q_i^*$ be the $i$th row of $Q$ and denote the $j$th column of $A$ by $a_j$ as usual.  Note that a row of a unitary matrix has unit $2$-norm.  By the Cauchy-Schwarz inequality,
\begin{align*}
|\tilde f(A)_{ij} - f(A)_{ij}| &= |\tilde g(a_j) - g(a_j)| = |q_i^* (a_j + \delta a_j) - q_i^* a_j|\\
  &= |q_i^* \delta a_j| \le \|q_i^*\|_2 \|\delta a_j\|_2 = \|\delta a_j\|_2 \le c(m) \ema \|a_j\|_2.
\end{align*}
In this calculation ``$\delta a_j$'' actually varies with (depends on) both $i$ and $j$, but the final bound is independent of $i$.

This entry-wise bound can be advanced to a Frobenius norm bound.  That is,
\begin{align*}
\|\tilde f(A) - f(A)\|_F^2 &= \sum_{\begin{smallmatrix} i=1,\dots,m \\ j=1,\dots,n \end{smallmatrix}}|\tilde f(A)_{ij} - f(A)_{ij}|^2 \le \sum_{i,j} c(m)^2 \ema^2 \|a_j\|_2^2 \\
   &= m\, c(m)^2 \ema^2 \sum_{j} \|a_j\|_2^2 = m\, c(m)^2 \ema^2 \|A\|_F^2.
\end{align*}
(The sum over $i$ gives the factor of $m$.  Note that $\sum_{j=1}^n \|a_j\|_2^2 = \|A\|_F^2$.)  Thus
    $$\|\tilde f(A) - f(A)\|_F \le \sqrt{m}\, c(m) \ema \|A\|_F.$$

Now we change tack and describe the forward error as a backward error.  Let
    $$\delta A = Q^*(\tilde f(A) - f(A))$$
so that $Q \delta A = \tilde f(A) - f(A)$.  Observe that
    $$\tilde f(A) = \tilde f(A) - f(A) + f(A) = Q \delta A + QA = Q (A+\delta A).$$
Let $\tilde A=A+\delta A$.  We have
    $$\tilde f(A) = f(\tilde A).$$

We now show that the backward error $\|\tilde A - A\|_F$ is relatively small by using the unitary invariance of the Frobenius norm:
\begin{align*}
\frac{\|\tilde A - A\|_F}{\|A\|_F} &= \frac{\|\delta A\|_F}{\|A\|_F} = \frac{\|Q\delta A\|_F}{\|A\|_F} = \frac{\|\tilde f(A) - f(A)\|_F}{\|A\|_F} \\
   &\le \frac{\sqrt{m} c(m) \ema \|A\|_F}{\|A\|_F} = \sqrt{m}\, c(m) \ema.
\end{align*}
\end{proof}

\end{document}


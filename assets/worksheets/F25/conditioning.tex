\documentclass[11pt]{amsart}
%prepared in AMSLaTeX, under LaTeX2e
\addtolength{\oddsidemargin}{-.9in} 
\addtolength{\evensidemargin}{-.9in}
\addtolength{\topmargin}{-.9in}
\addtolength{\textwidth}{1.5in}
\addtolength{\textheight}{1.5in}

\renewcommand{\baselinestretch}{1.05}

\usepackage{verbatim} % for "comment" environment

\usepackage{palatino}

\usepackage[final]{graphicx}

\usepackage{tikz}
\usetikzlibrary{positioning}

\usepackage{enumitem,xspace,fancyvrb}

\newtheorem*{thm}{Theorem}
\newtheorem*{defn}{Definition}
\newtheorem*{example}{Example}
\newtheorem*{problem}{Problem}
\newtheorem*{remark}{Remark}

\DefineVerbatimEnvironment{mVerb}{Verbatim}{numbersep=2mm,frame=lines,framerule=0.1mm,framesep=2mm,xleftmargin=4mm,fontsize=\footnotesize}

% macros
\usepackage{amssymb}
\newcommand{\bA}{\mathbf{A}}
\newcommand{\bB}{\mathbf{B}}
\newcommand{\bE}{\mathbf{E}}
\newcommand{\bF}{\mathbf{F}}
\newcommand{\bJ}{\mathbf{J}}

\newcommand{\bb}{\mathbf{b}}
\newcommand{\br}{\mathbf{r}}
\newcommand{\bv}{\mathbf{v}}
\newcommand{\bw}{\mathbf{w}}
\newcommand{\bx}{\mathbf{x}}

\newcommand{\hbi}{\mathbf{\hat i}}
\newcommand{\hbj}{\mathbf{\hat j}}
\newcommand{\hbk}{\mathbf{\hat k}}
\newcommand{\hbn}{\mathbf{\hat n}}
\newcommand{\hbr}{\mathbf{\hat r}}
\newcommand{\hbt}{\mathbf{\hat t}}
\newcommand{\hbx}{\mathbf{\hat x}}
\newcommand{\hby}{\mathbf{\hat y}}
\newcommand{\hbz}{\mathbf{\hat z}}
\newcommand{\hbphi}{\mathbf{\hat \phi}}
\newcommand{\hbtheta}{\mathbf{\hat \theta}}
\newcommand{\complex}{\mathbb{C}}
\newcommand{\ppr}[1]{\frac{\partial #1}{\partial r}}
\newcommand{\ppt}[1]{\frac{\partial #1}{\partial t}}
\newcommand{\ppx}[1]{\frac{\partial #1}{\partial x}}
\newcommand{\ppy}[1]{\frac{\partial #1}{\partial y}}
\newcommand{\ppz}[1]{\frac{\partial #1}{\partial z}}
\newcommand{\pptheta}[1]{\frac{\partial #1}{\partial \theta}}
\newcommand{\ppphi}[1]{\frac{\partial #1}{\partial \phi}}
\newcommand{\Div}{\ensuremath{\nabla\cdot}}
\newcommand{\Curl}{\ensuremath{\nabla\times}}
\newcommand{\curl}[3]{\ensuremath{\begin{vmatrix} \hbi & \hbj & \hbk \\ \partial_x & \partial_y & \partial_z \\ #1 & #2 & #3 \end{vmatrix}}}
\newcommand{\cross}[6]{\ensuremath{\begin{vmatrix} \hbi & \hbj & \hbk \\ #1 & #2 & #3 \\ #4 & #5 & #6 \end{vmatrix}}}
\newcommand{\eps}{\epsilon}
\newcommand{\grad}{\nabla}
\newcommand{\ip}[2]{\ensuremath{\left<#1,#2\right>}}
\newcommand{\lam}{\lambda}
\newcommand{\lap}{\triangle}

\newcommand{\Null}{\operatorname{null}}
\newcommand{\rank}{\operatorname{rank}}
\newcommand{\range}{\operatorname{range}}
\newcommand{\trace}{\operatorname{tr}}

\newcommand{\RR}{\mathbb{R}}
\newcommand{\ZZ}{\mathbb{Z}}

\newcommand{\prob}[1]{\bigskip\noindent\textbf{#1.}\quad }
\newcommand{\exer}[2]{\prob{Exercise #2 on page #1}}
\newcommand{\exerpages}[2]{\prob{Exercise #2 on pages #1}}

\newcommand{\pts}[1]{(\emph{#1 pts}) }
\newcommand{\epart}[1]{\bigskip\noindent\textbf{(#1)}\quad }
\newcommand{\ppart}[1]{\,\textbf{(#1)}\quad }

\newcommand{\Matlab}{\textsc{Matlab}\xspace}


\begin{document}
\scriptsize \noindent Math 614 Numerical Linear Algebra (Bueler) \hfill \fbox{\emph{Not to be turned in!}}
\normalsize\medskip

\Large\centerline{\textbf{Worksheet: Computing condition numbers.}}
\medskip
\normalsize

\thispagestyle{empty}

\begin{quote}
The goal of this worksheet is to demystify condition numbers.  Please use the Lecture 12 formulas below, and your knowledge of norms (Lecture 3), and please refer to the text as needed.  Numbers are assumed real merely for conceptual simplicity.
\end{quote}

\bigskip
\noindent \textbf{Formulas.}  A \emph{problem} is a function $f:X \to Y$, where $X$ and $Y$ are normed vector spaces.  The \emph{Jacobian matrix} of a problem $f$ is its first derivative.  Specifically, if $X=\RR^n$ and $Y=\RR^m$ then the $i$th component of the output is $f_i(x)=f_i(x_1,\dots,x_n)$ and the Jacobian is an $m\times n$ matrix:
\begin{equation}
J = J_f(x) = \begin{bmatrix} \frac{\partial f_1}{\partial x_1} & \dots & \frac{\partial f_1}{\partial x_n} \\ \vdots & \ddots & \vdots \\ \frac{\partial f_m}{\partial x_1} & \dots & \frac{\partial f_m}{\partial x_n} \end{bmatrix}
\end{equation}
Condition numbers measure the sensitivity of problems by comparing output changes to input changes, in the limit of small input changes.  The \emph{absolute condition number} of a problem $f$ is defined as $\hat \kappa_f(x) = \sup_{\delta x} \|\delta f\|/\|\delta x\|$, and when $f$ has a derivative one may compute it as
\begin{equation}
\hat \kappa = \|J(x)\|.
\end{equation}
The (more) \emph{relative condition number} is $\kappa_f(x) = \sup_{\delta x} (\|\delta f\|/\|f(x)\|) \,/\, (\|\delta x\|/\|x\|)$, thus
\begin{equation}
\kappa = \frac{\|J(x)\|}{\|f(x)\|/\|x\|}.
\end{equation}

\bigskip
\noindent \textbf{Exercises.}  For each of the 4 problems below, please use formulas (1), (2), and (3) to compute $J$, $\hat \kappa$, and $\kappa$.  When you have a choice of norms, choose (and identify) the most convenient one.

\bigskip
\begin{enumerate}[leftmargin=15pt]
\renewcommand{\labelenumi}{\Large \textbf{\arabic{enumi}.}}
\item $f: (0,\infty) \to \RR$ has formula $f(x) = 1/x$.

\vfill
\item $f: \RR^2 \to \RR$ has formula $f(x) = x_1 + x_2$.

\vfill
\newpage
\item $f: \RR^n \to \RR$ has formula $f(x) = x_1^2 + x_2^2 + \dots + x_n^2$.
\vfill

\item $f: \RR^n \to \RR^m$ has formula $f(x) = A x$ where $A$ is a fixed $m\times n$ matrix.

\vfill
\end{enumerate}

\end{document}

\documentclass[11pt]{amsart}
%prepared in AMSLaTeX, under LaTeX2e
\addtolength{\oddsidemargin}{-.9in} 
\addtolength{\evensidemargin}{-.9in}
\addtolength{\topmargin}{-.9in}
\addtolength{\textwidth}{1.5in}
\addtolength{\textheight}{1.5in}

\renewcommand{\baselinestretch}{1.05}

\usepackage{verbatim} % for "comment" environment

\usepackage{palatino,enumitem}

\usepackage[final]{graphicx}

\usepackage{tikz}
\usetikzlibrary{positioning}

\usepackage{enumitem,xspace,fancyvrb}

\newtheorem*{thm}{Theorem}
\newtheorem*{defn}{Definition}
\newtheorem*{example}{Example}
\newtheorem*{problem}{Problem}
\newtheorem*{remark}{Remark}

\DefineVerbatimEnvironment{mVerb}{Verbatim}{numbersep=2mm,frame=lines,framerule=0.1mm,framesep=2mm,xleftmargin=4mm,fontsize=\footnotesize}

% macros
\usepackage{amssymb}
\newcommand{\bA}{\mathbf{A}}
\newcommand{\bB}{\mathbf{B}}
\newcommand{\bE}{\mathbf{E}}
\newcommand{\bF}{\mathbf{F}}
\newcommand{\bJ}{\mathbf{J}}

\newcommand{\bb}{\mathbf{b}}
\newcommand{\br}{\mathbf{r}}
\newcommand{\bv}{\mathbf{v}}
\newcommand{\bw}{\mathbf{w}}
\newcommand{\bx}{\mathbf{x}}

\newcommand{\CC}{\mathbb{C}}
\newcommand{\RR}{\mathbb{R}}

\newcommand{\Div}{\ensuremath{\nabla\cdot}}
\newcommand{\Curl}{\ensuremath{\nabla\times}}

\newcommand{\eps}{\epsilon}
\newcommand{\grad}{\nabla}
\newcommand{\image}{\operatorname{im}}

\newcommand{\ip}[2]{\ensuremath{\left<#1,#2\right>}}
\newcommand{\lam}{\lambda}

\newcommand{\Null}{\operatorname{null}}
\newcommand{\rank}{\operatorname{rank}}
\newcommand{\range}{\operatorname{range}}
\newcommand{\trace}{\operatorname{tr}}

\newcommand{\prob}[1]{\bigskip\noindent\textbf{#1.}\quad }
\newcommand{\exer}[2]{\prob{Exercise #2 on page #1}}
\newcommand{\exerpages}[2]{\prob{Exercise #2 on pages #1}}

\newcommand{\pts}[1]{(\emph{#1 pts}) }
\newcommand{\epart}[1]{\bigskip\noindent\textbf{(#1)}\quad }
\newcommand{\ppart}[1]{\,\textbf{(#1)}\quad }


\begin{document}
\scriptsize \noindent Math 614 Numerical Linear Algebra \hfill Bueler
\normalsize\medskip

\Large\centerline{\textbf{Worksheet: Recalling and using matrix factorizations.}}
\medskip
\normalsize

\thispagestyle{empty}

\begin{quote}
Coming just before Midterm Quiz 1, the goal of this worksheet is to recall \emph{which matrix factorizations are known already}, and \emph{how they are used to solve linear systems} and \emph{how they are used to build orthogonal projections}.
\end{quote}

\bigskip
\begin{enumerate}[leftmargin=-1mm]
\renewcommand{\labelenumi}{\textbf{\arabic{enumi}.}}
\item Assume $A\in\CC^{m\times m}$ is square and invertible.  State the \textbf{full SVD} of $A$, including properties of the factors, and then give the additional three steps needed to solve a linear system $Ax=b$.  Make notes on the cost of the steps.

\vfill

\item Assume $A\in\CC^{m\times m}$ is square, invertible, \emph{and diagonalizable}.  State a \textbf{diagonalization factorization} of $A$, including properties of the factors, and then give the additional three steps needed to solve a linear system $Ax=b$.

\vfill

\clearpage\newpage
\item Assume $A\in\CC^{m\times m}$ is square and invertible.  State the \textbf{full QR factorization} of $A$, including properties of the factors, and then give the additional two steps needed to solve a linear system $Ax=b$.  Make notes on the cost of the steps.

\vfill

\end{enumerate}

\noindent\hrulefill

\medskip
\begin{enumerate}[leftmargin=-1mm]
\renewcommand{\labelenumi}{\textbf{\Alph{enumi}.}}
\item Assume $A\in\CC^{m\times n}$, $m\ge n$, is full-rank.  State the \textbf{reduced SVD} of $A$, including sizes and properties of the factors.  Then give a formula for the projection onto the range of $A$.

\vfill

\item Assume $A\in\CC^{m\times n}$, $m\ge n$, is full-rank.  State the \textbf{reduced QR factorization} of $A$, including sizes and properties of the factors.  What is the cost of this factorization?  Then give a formula for the projection onto the range of $A$.

\vfill

\end{enumerate}

\end{document}
